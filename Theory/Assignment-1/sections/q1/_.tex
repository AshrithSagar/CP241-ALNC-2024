\section*{Problem 1}

``Linear Systems Primer'' by Antsaklis and Michael;
Exercise 1.1(a)

\(M_1\) and \(M_2\) denote point masses; \(K_1, K_2, K\) denote spring constants; and \(x_1,x_2\) denote the displacements of the masses \(M_1\) and \(M_2\).
Use the Hamiltonian formulation of dynamical systems described above to derive a system of first-order ordinary differential equations that characterize this system.
Verify your answer by using Newton's second law of motion to derive the same system of equations.
Assume that \(x_1(0), \dot x_1(0), x_2(0)\), and \(\dot x_2(0)\) are given.

\begin{figure}[h]
    \centering
    \includegraphics[width=0.7\textwidth]{figures/images/q1.png}
    \caption{
        Example of a conservative dynamical system
    }\label{fig:q1}
\end{figure}

\textit{Hint}: Use \(p\) and \(q\) as your state variables in both cases.
For derivation using Newton's law, mention the Free Body Diagram with all the forces and the direction taken as +ve direction.

\subsection*{Solution}
