\section*{Problem 3}

A dynamical system is given by the following state-space and output equations with nominal input \( u = 0\):
\begin{align*}
    \begin{bmatrix}
        \dot{x}_1 \\
        \dot{x}_2 \\
        \dot{x}_3
    \end{bmatrix}
      & =
    \begin{bmatrix}
        x_2 - 2 x_1^2    \\
        x_3 + 3 x_1 x_2  \\
        -x_1 - x_2^2 + u \\
    \end{bmatrix}
    \\
    y & = x_1^2 + x_2
\end{align*}

\begin{enumerate}[label= (\alph*)]
    \item Compute the equilibrium point(s) under zero input conditions.
          Note that, at the equilibrium point, system behavior does not change over time.
    \item Find the linearized system around all the equilibrium points.
\end{enumerate}

\subsection*{Solution}

\subsubsection*{(a) Equilibrium point(s) under zero input conditions}

We have that for the equilibrium points, system behavior doesn't change over time, i.e., the time derivatives of the state variables are zero, i.e., \( \dot x_1 = 0, \ \dot x_2 = 0, \ \dot x_3 = 0 \).

Thus, we have, under zero input conditions with \( u = 0 \),
\begin{align*}
    x_2 - 2 x_1^2   & = 0 \\
    x_3 + 3 x_1 x_2 & = 0 \\
    -x_1 - x_2^2    & = 0
\end{align*}

Solving the above equations, we get the following two real solutions for \( x_1 \):
\[
    x_1(1 + 4x_1^3) = 0
    \implies
    x_1 \in \{ 0, \ -2^{-2/3} \}
\]

Hence, we get the equilibrium points as
\[
    \boxed{
        \mathbf{x^*} =
        \Bigg \{
        \begin{bmatrix}
            0 \\
            0 \\
            0
        \end{bmatrix},
        \begin{bmatrix}
            -2^{-2/3} \\
            2^{-1/3}  \\
            3/2
        \end{bmatrix}
        \Bigg \}
    }
\]

\subsubsection*{(b) Linearized system around all the equilibrium points}

Given a non-linear system of the form \( \mathbf{\dot x}(t) = \mathbf{f}(x_1, x_2, x_3) \), we can linearise it around the equilibrium point \( \mathbf{x^*} \) as follows:
\[
    \mathbf{\dot x}(t) = \mathbf{f}(x_1, x_2, x_3) \approx \mathbf{f}(\mathbf{x^*}) + \mathbf{f'}(\mathbf{x^*})\; (\mathbf{x} - \mathbf{x^*}) + \text{H.O.T.}
\]
where \(\mathbf{f'}\) denotes the Jacobian matrix of \(\mathbf{f}\), and \text{H.O.T.} denotes higher-order terms.

At the equilibrium points, we have
\[
    \mathbf{f}(\mathbf{x^*}) =
    \begin{bmatrix}
        0 \\
        0 \\
        0
    \end{bmatrix}
\]

Now, for the equilibrium point \( \mathbf{x^*} = \begin{bmatrix} 0 \\ 0 \\ 0 \end{bmatrix} \), we have the Jacobian matrix of the system as
\[
    \mathbf{f'}(\mathbf{x}) =
    \begin{bmatrix}
        -4x_1 & 1     & 0 \\
        3x_2  & 3x_1  & 1 \\
        -1    & -2x_2 & 0
    \end{bmatrix}
    \implies
    \mathbf{f'}(\mathbf{x^*}) =
    \begin{bmatrix}
        0  & 1 & 0 \\
        0  & 0 & 1 \\
        -1 & 0 & 0
    \end{bmatrix}
\]

Thereby, we get
\[
    \mathbf{\dot x}(t) \approx
    \begin{bmatrix}
        0  & 1 & 0 \\
        0  & 0 & 1 \\
        -1 & 0 & 0
    \end{bmatrix}
    \begin{bmatrix}
        x_1(t) \\
        x_2(t) \\
        x_3(t)
    \end{bmatrix}
    \implies
    \boxed{
        \mathbf{\dot x}(t) \approx
        \begin{bmatrix}
            x_2(t) \\
            x_3(t) \\
            -x_1(t)
        \end{bmatrix}
    }
\]

For the equilibrium point \( \mathbf{x^*} = \begin{bmatrix} -2^{-2/3} \\ 2^{-1/3}  \\ 3/2 \end{bmatrix} \), we have the Jacobian matrix of the system as
\[
    \mathbf{f'}(\mathbf{x}) =
    \begin{bmatrix}
        -4x_1 & 1     & 0 \\
        3x_2  & 3x_1  & 1 \\
        -1    & -2x_2 & 0
    \end{bmatrix}
    \implies
    \mathbf{f'}(\mathbf{x^*}) =
    \begin{bmatrix}
        2^{4/3}         & 1                & 0 \\
        3\cdot 2^{-1/3} & -3\cdot 2^{-2/3} & 1 \\
        -1              & -2^{2/3}         & 0
    \end{bmatrix}
\]

Thereby, we get
\begin{align*}
    \mathbf{\dot x}(t)
     & \approx
    \begin{bmatrix}
        2^{4/3}         & 1                & 0 \\
        3\cdot 2^{-1/3} & -3\cdot 2^{-2/3} & 1 \\
        -1              & -2^{2/3}         & 0
    \end{bmatrix}
    \begin{bmatrix}
        x_1(t) + 2^{-2/3} \\
        x_2(t) - 2^{-1/3} \\
        x_3(t) - 3/2
    \end{bmatrix} \\
     & =
    \begin{bmatrix}
        2^{4/3}x_1(t) + 2^{4/3-2/3} + x_2(t) - 2^{-1/3}                                                          \\
        3\cdot 2^{-1/3}x_1(t) + 3\cdot 2^{-1/3-2/3} - 3\cdot 2^{-2/3}x_2(t) + 3\cdot 2^{-2/3-1/3} + x_3(t) - 3/2 \\
        -x_1(t) - 2^{-2/3} - 2^{2/3}x_2(t) + 2^{2/3-1/3}
    \end{bmatrix}
\end{align*}
\[
    \boxed{
        \mathbf{\dot x}(t) \approx
        \begin{bmatrix}
            2^{4/3} x_1(t) + x_2(t) + 2^{-1/3}                             \\
            3\cdot 2^{-1/3} x_1(t) - 3\cdot 2^{-2/3} x_2(t) + x_3(t) + 3/2 \\
            - x_1(t) - 2^{2/3} x_2(t) + 2^{-2/3}
        \end{bmatrix}
    }
\]
\[
    \mathbf{\dot x}(t) \approx
    \begin{bmatrix}
        2.519\, x_1(t) + x_2(t) + 0.7937                 \\
        2.3811\, x_1(t) - 1.8899\, x_2(t) + x_3(t) + 0.5 \\
        - x_1(t) - 1.5874\, x_2(t) + 0.63
    \end{bmatrix}
\]
