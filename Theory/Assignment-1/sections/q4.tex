\section*{Problem 4}

The steel ball is suspended in the air by the electromagnet as shown in Figure~\ref{fig:q4}.
The objective is to keep the ball suspended at a particular equilibrium position by controlling the current \(i(t)\) applying proper input voltage \(v(t)\), where \(t \geq 0\) denotes time.
The resistance of the coil is \(R\) and the Inductance of the coil is given by \(L(s(t)) = \frac{L}{s(t)}\), with \(L\) is a constant and \(s(t)\) is the distance between the magnet and center of the ball at time t.
The force produced by the magnet is given by \(\frac{K i^2(t)}{s^2(t)}\), where \(K\) is a constant and \(g\) is the acceleration due to gravity.

\begin{enumerate}[label= (\alph*)]
    \item Determine the differential equations governing the dynamics of the system.
    \item Let \(v(t) = V_{in}\) is a nominal voltage. \\
          Find out the resulting equilibrium point of the system.
    \item Obtain the linearized system around the derived equilibrium point.
\end{enumerate}

\begin{figure}[h]
    \centering
    \includegraphics[width=0.5\textwidth]{figures/images/q4.png}
    \caption{
        Magnetic Levitation System
    }\label{fig:q4}
\end{figure}

\clearpage
\subsection*{Solution}

\subsubsection*{(a) Governing differential equations}

Consider the upward direction to be the positive \(y\)-direction.
The underlying laws are the Newton's laws of motion and Kirchoff's voltage law.
The voltage across the coil is given by, with the magnetic flux as \( \Phi \),
\[
    v_L(t)
    = -\frac{\Phi}{dt}
    = -\frac{d\; \Big( L(s(t))\; i(t) \Big)}{dt}
    = -\frac{L}{s(t)} \frac{d}{dt} i(t)
    - i \frac{d}{dt} \frac{L}{s(t)}
    = -\frac{L}{s(t)} \frac{d}{dt} i(t)
    + \frac{Li(t)}{s^2(t)}
\]
We have the differential equations governing the dynamics of the system as follows,
\begin{align*}
    M \ddot s(t) & = \frac{Ki^2(t)}{s^2(t)} - Mg       \\
    v(t)         & = -\frac{L}{s(t)} \frac{d}{dt} i(t)
    + \frac{Li(t)}{s^2(t)} + R i(t)
\end{align*}

Taking \( x_1 = i, \ x_2 = s, \ x_3 = \dot s \), we can rewrite the above equations as
\[
    \boxed{
        \begin{aligned}
            \dot x_1(t) & = \frac{-1}{L}\, x_2(t)\, \Big[ v(t) - R\, x_1(t) - L \frac{x_1(t)}{x_2^2(t)} \Big] \\
            \dot x_2(t) & = x_3(t)                                                                            \\
            \dot x_3(t) & = \frac{K}{M} \frac{x_1^2(t)}{x_2^2(t)} - g
        \end{aligned}
    }
\]

\subsubsection*{(b) Equilibrium point(s) under input \(v(t)=V_{in}\)}

We have that for the equilibrium points, system behavior doesn't change over time, i.e., the time derivatives of the state variables are zero, i.e., \( \dot x_1(t) = 0, \ \dot x_2(t) = 0, \ \dot x_3(t) = 0 \).

Thereby, under the input conditions \( v(t)=V_{in} \), we get
\begin{align*}
    x_1(t) & = \frac{V_{in}}{R}                                                  \\
    x_3(t) & = 0                                                                 \\
    x_2(t) & = x_1(t) \sqrt{\frac{K}{Mg}} = \frac{V_{in}}{R} \sqrt{\frac{K}{Mg}}
\end{align*}

Note that \( x_2(t) = 0 \) causes a singularity in the \( \dot x_3(t) \) equation.

Thus, the equilibrium points of the system are
\[
    \boxed{
        \mathbf{x^*} =
        \begin{bmatrix}
            \cfrac{V_{in}}{R}                      \\
            \\
            \cfrac{V_{in}}{R} \sqrt{\cfrac{K}{Mg}} \\
            \\
            0
        \end{bmatrix}
    }
\]

We can observe that at the equilibrium point that both \( \dot s(t) \) and \( \ddot s(t) \) are zero.

\subsubsection*{(c) Linearized system around all the equilibrium points}

Given a non-linear system of the form \( \mathbf{\dot x}(t) = \mathbf{f}(x_1, x_2, x_3) \), we can linearise it around the equilibrium point \( \mathbf{x^*} \) as follows:
\[
    \mathbf{\dot x}(t) = \mathbf{f}(x_1, x_2, x_3) \approx \mathbf{f}(\mathbf{x^*}) + \mathbf{f'}(\mathbf{x^*})\; (\mathbf{x} - \mathbf{x^*}) + \text{H.O.T.}
\]
where \(\mathbf{f'}\) denotes the Jacobian matrix of \(\mathbf{f}\), and \text{H.O.T.} denotes higher-order terms.

At the equilibrium points, we have
\[
    \mathbf{f}(\mathbf{x^*}) =
    \begin{bmatrix}
        0 \\
        0 \\
        0
    \end{bmatrix}
\]

The Jacobian matrix of the system is
\[
    \mathbf{f'}(\mathbf{x}) =
    \begin{bmatrix}
        \cfrac{\partial f_1}{\partial x_1} &  &  & \cfrac{\partial f_1}{\partial x_2} &  &  & \cfrac{\partial f_1}{\partial x_3} \\
        \\
        \cfrac{\partial f_2}{\partial x_1} &  &  & \cfrac{\partial f_2}{\partial x_2} &  &  & \cfrac{\partial f_2}{\partial x_3} \\
        \\
        \cfrac{\partial f_3}{\partial x_1} &  &  & \cfrac{\partial f_3}{\partial x_2} &  &  & \cfrac{\partial f_3}{\partial x_3} \\
    \end{bmatrix}
\]
\[
    \implies
    \mathbf{f'}(\mathbf{x}) =
    \begin{bmatrix}
        \cfrac{-R}{L}x_2(t)                    &  & \cfrac{V_{in} - R x_1(t)}{L}              &  & 0 \\
        \\
        0                                      &  & 0                                         &  & 1 \\
        \\
        \cfrac{2K}{M} \cfrac{x_1(t)}{x_2^2(t)} &  & \cfrac{-2K}{M} \cfrac{x_1^2(t)}{x_2^3(t)} &  & 0
    \end{bmatrix}
\]
\[
    \implies
    \mathbf{f'}(\mathbf{x^*}) =
    \begin{bmatrix}
        \cfrac{-V_{in}}{L} \sqrt{\cfrac{K}{Mg}} &  & 0                                        &  & 0 \\
        0                                       &  & 0                                        &  & 1 \\
        \cfrac{2 R g}{V_{in}}                   &  & \cfrac{-2gR}{V_{in}}\sqrt{\cfrac{Mg}{K}} &  & 0
    \end{bmatrix}
\]

Thereby, we get
\[
    \mathbf{\dot x}(t) \approx
    \begin{bmatrix}
        \cfrac{-V_{in}}{L} \sqrt{\cfrac{K}{Mg}} &  & 0                                        &  & 0 \\
        0                                       &  & 0                                        &  & 1 \\
        \cfrac{2 R g}{V_{in}}                   &  & \cfrac{-2gR}{V_{in}}\sqrt{\cfrac{Mg}{K}} &  & 0
    \end{bmatrix}
    \begin{bmatrix}
        x_1(t) - \cfrac{V_{in}}{R}                      \\
        \\
        x_2(t) - \cfrac{V_{in}}{R} \sqrt{\cfrac{K}{Mg}} \\
        \\
        x_3(t)
    \end{bmatrix}
\]
\[
    \implies
    \mathbf{\dot x}(t) \approx
    \begin{bmatrix}
        \cfrac{-V_{in}}{L} \sqrt{\cfrac{K}{Mg}} \Bigg( x_1(t) - \cfrac{V_{in}}{R} \Bigg) \\
        x_3(t)                                                                           \\
        \cfrac{2 R g}{V_{in}} \Bigg( x_1(t) - \cfrac{V_{in}}{R} \Bigg) + \cfrac{-2gR}{V_{in}}\sqrt{\cfrac{Mg}{K}} \Bigg( x_2(t) - \cfrac{V_{in}}{R} \sqrt{\cfrac{K}{Mg}} \Bigg)
    \end{bmatrix}
\]

Therefore, the linearised system around the equilibrium point \( \mathbf{x^*} \) is
\[
    \therefore
    \boxed{
        \mathbf{\dot x}(t) \approx
        \begin{bmatrix}
            \cfrac{-V_{in}}{L} \sqrt{\cfrac{K}{Mg}} x_1(t) + \cfrac{V_{in}^2}{LR} \sqrt{\cfrac{K}{Mg}} \\
            x_3(t)                                                                                     \\
            \cfrac{2 R g}{V_{in}} x_1(t) + \cfrac{-2gR}{V_{in}} \sqrt{\cfrac{Mg}{K}} x_2(t) + 4g
        \end{bmatrix}
    }
\]
