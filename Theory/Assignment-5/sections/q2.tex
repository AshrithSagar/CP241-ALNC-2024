\section*{Problem 2}

Consider the system shown in Figure~\ref{fig:q2}.
It consists of one block with mass \( m \), connected by two springs with spring constants \( k_{1}, k_{2} \).
The input \( u=F \) is applied in both horizontal and vertical directions.
\begin{enumerate}[label= (\alph*), topsep=3pt, itemsep=-0.2em]
    \item Taking the displacements \( x, y \) and velocities \( \dot{x}, \dot{y} \) as state variables, derive the state-space equations.
    \item Is this system controllable if (i) \( k_{1} \neq k_{2} \) and (ii) \( k_{1}=k_{2} \)?
\end{enumerate}
\begin{figure}[h]
    \centering
    \includegraphics[width=0.3\linewidth]{figures/images/q2.jpg}
    \caption{
        Spring mass system
    }\label{fig:q2}
\end{figure}
\vspace*{-2.5em}

\subsection*{Solution}

\subsubsection*{(a) State-space equations}

Assuming the acceleration due to gravity \( g \) is zero.

We can consider the block with mass \( m \) as a point mass, and can decompose the motion in two orthogonal directions, as shown in Figure~\ref{fig:q2-decomposition}.
We make an assumption that each of the springs act independently of each other, so that the horizontal and vertical motion can be analyzed separately.

\begin{figure}[h]
    \centering
    \includegraphics[width=\linewidth]{figures/q2/_}
    \caption{
        Free body diagram of the spring mass system
    }\label{fig:q2-decomposition}
\end{figure}

The forces exerted by the springs are given by Hooke's law, as
\begin{align*}
    F_{1}
     & = -k_{1}x
    \\
    F_{2}
     & = -k_{2}y
\end{align*}

The equations of motion are given by Newton's second law, as
\begin{align*}
    m\ddot{x}
     & = -k_{1}x + F_{x}
    \\
    m\ddot{y}
     & = -k_{2}y + F_{y}
\end{align*}
