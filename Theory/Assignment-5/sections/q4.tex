\section*{Problem 4}

For the following systems, use linearizations to design a state feedback controller to stabilize the origin.
\begin{enumerate}[label= (\alph*), topsep=3pt, itemsep=-1em]
    \item \begin{align*}
               & \dot{x_{1}}=x_{1}+x_{2}               \\
               & \dot{x_{2}}=3 x_{1}^{2} x_{2}+x_{1}+u
          \end{align*}
          \vspace*{-1.5em}

    \item \begin{align*}
              \dot{x_{1}} & =x_{1}+x_{2}                 \\
              \dot{x_{2}} & =x_{1} x_{2}^{2}+x_{1}+x_{3} \\
              \dot{x_{3}} & =u
          \end{align*}
\end{enumerate}
\vspace*{-3em}

\subsection*{Solution}

\subsubsection*{(a)}

We will first linearize the system around the origin.\\
The Jacobian matrix of the system is given by
\begin{align*}
    J(x)
     & =
    \begin{bmatrix}
        \frac{\partial f_{1}}{\partial x_{1}} & \frac{\partial f_{1}}{\partial x_{2}} \\
        \frac{\partial f_{2}}{\partial x_{1}} & \frac{\partial f_{2}}{\partial x_{2}}
    \end{bmatrix}
    =
    \begin{bmatrix}
        1               & 1           \\
        6 x_{1} x_{2}+1 & 3 x_{1}^{2}
    \end{bmatrix}
    \implies
    A
    =
    J(0)
    =
    \begin{bmatrix}
        1 & 1 \\
        1 & 0
    \end{bmatrix}
\end{align*}
Thereby, the linearized system is given by
\begin{align*}
    \dot{x}
     & =
    \begin{bmatrix}
        1 & 1 \\
        1 & 0
    \end{bmatrix}
    x
    +
    \begin{bmatrix}
        0 \\
        1
    \end{bmatrix}
    u
    \implies
    \dot x = Ax + Bu
\end{align*}
The state feedback gain \( K \) is given by
\(
K =
\begin{bmatrix}
    k_{1} & k_{2}
\end{bmatrix}
\)
and we have
\begin{equation*}
    A - BK
    =
    \begin{bmatrix}
        1 & 1 \\
        1 & 0
    \end{bmatrix}
    -
    \begin{bmatrix}
        0 \\ 1
    \end{bmatrix}
    \begin{bmatrix}
        k_{1} & k_{2}
    \end{bmatrix}
    =
    \begin{bmatrix}
        1 & 1 \\
        1 & 0
    \end{bmatrix}
    -
    \begin{bmatrix}
        0     & 0     \\
        k_{1} & k_{2}
    \end{bmatrix}
    =
    \begin{bmatrix}
        1       & 1    \\
        1 - k_1 & -k_2
    \end{bmatrix}
\end{equation*}
We check the controllability of the system by checking the rank of the controllability matrix.
\begin{align*}
    \mathcal{C}
     & =
    \begin{bmatrix}
        B & AB
    \end{bmatrix}
    =
    \begin{bmatrix}
        0 & 1 \\
        1 & 0
    \end{bmatrix}
    \implies
    \text{rank}(\mathcal{C}) = 2
\end{align*}
Thereby, the system is controllable.
Now we try to use a state feedback controller to stabilize the system at the origin.
The characteristic equation of the closed-loop system is given by
\begin{align*}
     &
    \det(sI-(A-BK))
    =
    \det
    \begin{bmatrix}
        s-1      & -1      \\
        -1+k_{1} & s+k_{2}
    \end{bmatrix}
    =
    0
    \\ &
    \implies
    s^2 - s(1-k_2) + (-k_2 + k_1 - 1) = 0
\end{align*}
The eigenvalues are given by
\begin{align*}
    s
     & =
    \frac{1-k_2}{2} \pm \frac{1}{2} \sqrt{{(1-k_2)}^2 - 4(-k_2 + k_1 - 1)}
    \\ & =
    \frac{1-k_2}{2} \pm \frac{1}{2} \sqrt{1 - 2k_2 + k_2^2 + 4k_2 - 4k_1 + 4}
    \\ & =
    \frac{1-k_2}{2} \pm \frac{1}{2} \sqrt{k_2^2 + 2k_2 - 4k_1 + 5}
\end{align*}
For the system to be stabilisable, the real parts of the eigenvalues must be negative.

By Routh-Hurwitz's stability criterion, we need the coefficients of the characteristic equation to be positive for the system to be stable.
Thereby, we get
\begin{align*}
    -(1-k_2)         & > 0 \implies k_2 < 1
    \\
    (-k_2 + k_1 - 1) & > 0 \implies k_1 - k_2 > 1
\end{align*}

\subsubsection*{(b)}

We will first linearize the system around the origin.\\
The Jacobian matrix of the system is given by
\begin{align*}
    J(x)
     & =
    \begin{bmatrix}
        \frac{\partial f_{1}}{\partial x_{1}} & \frac{\partial f_{1}}{\partial x_{2}} & \frac{\partial f_{1}}{\partial x_{3}} \\
        \frac{\partial f_{2}}{\partial x_{1}} & \frac{\partial f_{2}}{\partial x_{2}} & \frac{\partial f_{2}}{\partial x_{3}} \\
        \frac{\partial f_{3}}{\partial x_{1}} & \frac{\partial f_{3}}{\partial x_{2}} & \frac{\partial f_{3}}{\partial x_{3}}
    \end{bmatrix}
    =
    \begin{bmatrix}
        1       & 1       & 0 \\
        x_2^2+1 & 2x_1x_2 & 1 \\
        0       & 0       & 0
    \end{bmatrix}
    \implies
    A
    =
    J(0)
    =
    \begin{bmatrix}
        1 & 1 & 0 \\
        1 & 0 & 1 \\
        0 & 0 & 0
    \end{bmatrix}
\end{align*}
Thereby, the linearized system is given by
\begin{align*}
    \dot{x}
     & =
    \begin{bmatrix}
        1 & 1 & 0 \\
        1 & 0 & 1 \\
        0 & 0 & 0
    \end{bmatrix}
    x
    +
    \begin{bmatrix}
        0 \\
        0 \\
        1
    \end{bmatrix}
    u
    \implies
    \dot x = Ax + Bu
\end{align*}
The state feedback gain \( K \) is given by
\(
K =
\begin{bmatrix}
    k_{1} & k_{2} & k_{3}
\end{bmatrix}
\)
and we have
\begin{equation*}
    A - BK
    =
    A
    -
    \begin{bmatrix}
        0 \\ 0 \\ 1
    \end{bmatrix}
    \begin{bmatrix}
        k_{1} & k_{2} & k_{3}
    \end{bmatrix}
    =
    \begin{bmatrix}
        1 & 1 & 0 \\
        1 & 0 & 1 \\
        0 & 0 & 0
    \end{bmatrix}
    -
    \begin{bmatrix}
        0   & 0   & 0   \\
        0   & 0   & 0   \\
        k_1 & k_2 & k_3
    \end{bmatrix}
    =
    \begin{bmatrix}
        1    & 1    & 0    \\
        1    & 0    & 1    \\
        -k_3 & -k_1 & -k_2
    \end{bmatrix}
\end{equation*}
We check the controllability of the system by checking the rank of the controllability matrix.
\begin{align*}
    \mathcal{C}
     & =
    \begin{bmatrix}
        B & AB & A^{2}B
    \end{bmatrix}
    =
    \begin{bmatrix}
        0 & 0 & 1 \\
        0 & 1 & 0 \\
        1 & 0 & 0
    \end{bmatrix}
    \implies
    \text{rank}(\mathcal{C}) = 3
\end{align*}
Thereby, the system is controllable.
Now we try to use a state feedback controller to stabilize the system at the origin.
The characteristic equation of the closed-loop system is given by
\begin{align*}
     &
    \det(sI-(A-BK))
    =
    \det
    \begin{bmatrix}
        s-1   & -1    & 0       \\
        -1    & s     & -1      \\
        k_{3} & k_{1} & s+k_{2}
    \end{bmatrix}
    =
    0
    \\ &
    \implies
    (s-1)(s(s+k_2) + k_1) + (-s-k_2 + k_3) = 0
    \\ &
    \implies
    s^3 + s^2 (k_2 - 1) + s (k_1 - k_2 - 1) + (-k_1 - k_2 + k_3) = 0
\end{align*}

By Routh-Hurwitz's stability criterion, we need the coefficients of the characteristic equation to be positive for the system to be stable.
Thereby, we get
\begin{align*}
    k_2 - 1          & > 0 \implies k_2 > 1
    \\
    k_1 - k_2 - 1    & > 0 \implies k_1 > k_2 + 1
    \\
    -k_1 - k_2 + k_3 & > 0 \implies k_3 > k_1 + k_2
\end{align*}
