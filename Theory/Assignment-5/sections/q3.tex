\section*{Problem 3}

An inverted pendulum of mass \( m \), is hinged at A.
A gyro with spin angular momentum, \( h \), is attached to the pendulum but is free to rotate about the pendulum axis \( (\phi) \) as shown in Figure~\ref{fig:q3}.
A control torque, \( Q \), can be applied to the gyro from the pendulum.
The equations of motion are
\vspace*{-1em}
\begin{align*}
    I \ddot{\theta} & =m g l \theta-h \dot{\phi} \\
    J \ddot{\phi}   & =h \dot{\theta}+Q
\end{align*}

\vspace*{-0.5em}
where \( I= \) moment of inertia of the pendulum plus gyro about \( A, J= \) moment of inertia of the gyro about axis \( A C \), and \( C= \) mass centre of the pendulum plus gyro.
\begin{enumerate}[label= (\alph*), topsep=3pt, itemsep=-0.25em]
    \item Show that the system is controllable by \( Q \).
    \item Show that the system is always unstable.
\end{enumerate}
\vspace*{-1.5em}
\begin{figure}[h]
    \centering
    \includegraphics[width=0.2\linewidth]{figures/images/q3.jpg}
    \caption{
        Inverted pendulum
    }\label{fig:q3}
\end{figure}
\vspace*{-3em}

\subsection*{Solution}

We can form the state-space representation of the system by using \( x_1=\theta, x_2=\dot{\theta}, x_3=\phi, x_4=\dot{\phi} \).
The state-space representation of the system then, is
\begin{align*}
    \begin{bmatrix}
        \dot{x}_1 \\ \dot{x}_2 \\ \dot{x}_3 \\ \dot{x}_4
    \end{bmatrix}
     & =
    \begin{bmatrix}
        0               & 1           & 0 & 0            \\
        \frac{m g l}{I} & 0           & 0 & -\frac{h}{I} \\
        0               & 0           & 0 & 1            \\
        0               & \frac{h}{J} & 0 & 0
    \end{bmatrix}
    \begin{bmatrix}
        x_1 \\ x_2 \\ x_3 \\ x_4
    \end{bmatrix}
    +
    \begin{bmatrix}
        0 \\ 0 \\ 0 \\ \frac{1}{J}
    \end{bmatrix} Q
    \\
    \implies
    \dot X
     & =
    A X + B Q
\end{align*}

\clearpage
\subsubsection*{(a) Controllability of the system by \( Q \)}

The controllability matrix is given by
\begin{equation*}
    \mathcal{C}
    =
    \begin{bmatrix}
        B & AB & A^{2}B & A^{3}B
    \end{bmatrix}
\end{equation*}
\begin{align*}
    \implies
    AB
     & =
    \begin{bmatrix}
        0               & 1           & 0 & 0            \\
        \frac{m g l}{I} & 0           & 0 & -\frac{h}{I} \\
        0               & 0           & 0 & 1            \\
        0               & \frac{h}{J} & 0 & 0
    \end{bmatrix}
    \begin{bmatrix}
        0 \\ 0 \\ 0 \\ \frac{1}{J}
    \end{bmatrix}
    =
    \begin{bmatrix}
        0              \\
        - \frac{h}{IJ} \\
        \frac{1}{J}    \\
        0
    \end{bmatrix}
    \\
    \implies
    A^{2}B
     & =
    \begin{bmatrix}
        \frac{m g l}{I} & 0                              & 0 & -\frac{h}{I}    \\
        0               & \frac{m g l}{I}-\frac{h^2}{IJ} & 0 & 0               \\
        0               & \frac{h}{J}                    & 0 & 0               \\
        \frac{mglh}{IJ} & 0                              & 0 & -\frac{h^2}{IJ}
    \end{bmatrix}
    \begin{bmatrix}
        0 \\ 0 \\ 0 \\ \frac{1}{J}
    \end{bmatrix}
    =
    \begin{bmatrix}
        -\frac{h}{IJ} \\
        0             \\
        0             \\
        -\frac{h^2}{IJ^2}
    \end{bmatrix}
    \\
    \implies
    A^{3}B
     & =
    \begin{bmatrix}
        0                                               & \frac{m g l}{I}-\frac{h^2}{IJ}        & 0 & 0                                     \\
        \frac{m^2 g^2 l^2}{I^2}-\frac{m g l h^2}{I^2 J} & 0                                     & 0 & \frac{h^3}{I^2 J}-\frac{m g l h}{I^2} \\
        \frac{m g l h}{IJ}                              & 0                                     & 0 & -\frac{h^2}{IJ}                       \\
        0                                               & -\frac{h^3}{I J^2}+\frac{m g l h}{IJ} & 0 & 0
    \end{bmatrix}
    \begin{bmatrix}
        0 \\ 0 \\ 0 \\ \frac{1}{J}
    \end{bmatrix}
    =
    \begin{bmatrix}
        0                                         \\
        \frac{h^3}{I^2 J^2}-\frac{m g l h}{I^2 J} \\
        -\frac{h^2}{IJ^2}                         \\
        0
    \end{bmatrix}
    \\
    \implies
    \mathcal{C}
     & =
    \begin{bmatrix}
        0           & 0             & -\frac{h}{IJ}     & 0                                         \\
        0           & -\frac{h}{IJ} & 0                 & \frac{h^3}{I^2 J^2}-\frac{m g l h}{I^2 J} \\
        0           & \frac{1}{J}   & 0                 & -\frac{h^2}{IJ^2}                         \\
        \frac{1}{J} & 0             & -\frac{h^2}{IJ^2} & 0
    \end{bmatrix}
\end{align*}
For the system to be controllable, the rank of the controllability matrix should be equal to the number of states, i.e., it should be full rank.

We can compute the determinant of the controllability matrix as
\begin{align*}
    \text{det}(\mathcal{C})
     & =
    \begin{vmatrix}
        0           & 0             & -\frac{h}{IJ}     & 0                                         \\
        0           & -\frac{h}{IJ} & 0                 & \frac{h^3}{I^2 J^2}-\frac{m g l h}{I^2 J} \\
        0           & \frac{1}{J}   & 0                 & -\frac{h^2}{IJ^2}                         \\
        \frac{1}{J} & 0             & -\frac{h^2}{IJ^2} & 0
    \end{vmatrix}
    =
    -\frac{h}{IJ}
    \begin{vmatrix}
        0           & -\frac{h}{IJ} & \frac{h^3}{I^2 J^2}-\frac{m g l h}{I^2 J} \\
        0           & \frac{1}{J}   & -\frac{h^2}{IJ^2}                         \\
        \frac{1}{J} & 0             & 0
    \end{vmatrix}
    \\ & =
    \left( -\frac{h}{IJ} \right) \left[ \left( \frac{h}{IJ} \right) \left( \frac{h^2}{I J^3} \right) + \left( \frac{h^3}{I^2 J^2}-\frac{m g l h}{I^2 J} \right) \left( -\frac{1}{J^2} \right) \right]
    \\ & =
    \left( -\frac{h}{IJ} \right) \left[ \cancel{\frac{h^3}{I^2 J^4}} - \cancel{\frac{h^3}{I^2 J^4}} + \frac{m g l h}{I^2 J^3} \right]
    =
    -\frac{m g l h^2}{I^3 J^4}
\end{align*}

Since the determinant of the controllability matrix \( \det(\mathcal{C}) \) is non-zero in the general case for non-zero values of \( m, g, l, h, I, J \), thereby the controllability matrix is full rank.

Hence, the system is \underline{controllable} by \( Q \).
