\section*{Problem 1}

Consider the discrete-time state equation
\begin{align*}
    x[k+1]
     & =
    \begin{bmatrix}
        1 & 1 & -2 \\
        0 & 1 & 1  \\
        0 & 0 & 1
    \end{bmatrix}
    x[k] +
    \begin{bmatrix}
        1 \\
        0 \\
        1
    \end{bmatrix}
    u[k]
    \\
    y[k]
     & =
    \begin{bmatrix}
        2 & 0 & 0
    \end{bmatrix}
    x[k]
\end{align*}
\begin{enumerate}[label= (\alph*)]
    \item Find the state feedback gain so that the resulting system has all eigenvalues at \( z=0 \).
    \item Show that for any initial state, the zero-input response of the feedback system becomes identically zero for \( k \geq 3 \).
\end{enumerate}

\subsection*{Solution}

\subsubsection*{(a) State feedback gain for all eigenvalues at \( z = 0 \) for the resulting system}

Given the system, we have
\begin{align*}
    A
     & =
    \begin{bmatrix}
        1 & 1 & -2 \\
        0 & 1 & 1  \\
        0 & 0 & 1
    \end{bmatrix}
    ,
    B
    =
    \begin{bmatrix}
        1 \\
        0 \\
        1
    \end{bmatrix}
\end{align*}
The state feedback gain \( K \) is given by
\(
K =
\begin{bmatrix}
    k_1 & k_2 & k_3
\end{bmatrix}
\)
and we have
\begin{equation*}
    A - BK
    =
    \begin{bmatrix}
        1 & 1 & -2 \\
        0 & 1 & 1  \\
        0 & 0 & 1
    \end{bmatrix}
    -
    \begin{bmatrix}
        k_1 & k_2 & k_3 \\
        0   & 0   & 0   \\
        k_1 & k_2 & k_3
    \end{bmatrix}
    =
    \begin{bmatrix}
        1-k_1 & 1-k_2 & -2-k_3 \\
        0     & 1     & 1      \\
        -k_1  & -k_2  & 1-k_3
    \end{bmatrix}
\end{equation*}
The characteristic equation of the system is given by
\begin{align*}
     &
    \det(zI - (A - BK))
    =
    \begin{vmatrix}
        z - 1 + k_1 & -1 + k_2 & 2 + k_3     \\
        0           & z - 1    & -1          \\
        k_1         & k_2      & z - 1 + k_3
    \end{vmatrix}
    =
    0
    \\ &
    \implies
    z^3 + z^2 (k_1 + k_3 - 3) + z (k_2 - 4k_1 - 2k_3 + 3) + (4k_1 - k_2 + k_3 - 1) = 0
\end{align*}
For the system to have all eigenvalues at \( z = 0 \), all the coefficients of the characteristic equation must be zero.
Thus, we have
\begin{align*}
    k_1 + k_3 - 3         & = 0 \\
    k_2 - 4k_1 - 2k_3 + 3 & = 0 \\
    4k_1 - k_2 + k_3 - 1  & = 0
\end{align*}
\begin{align*}
    \implies
    \begin{bmatrix}
        1  & 0  & 1  \\
        -4 & 1  & -2 \\
        4  & -1 & 1
    \end{bmatrix}
    \begin{bmatrix}
        k_1 \\
        k_2 \\
        k_3
    \end{bmatrix}
     & =
    \begin{bmatrix}
        3  \\
        -3 \\
        1
    \end{bmatrix}
    \implies
    \begin{bmatrix}
        k_1 \\
        k_2 \\
        k_3
    \end{bmatrix}
    =
    \begin{bmatrix}
        1  & 0  & 1  \\
        -4 & 1  & -2 \\
        4  & -1 & 1
    \end{bmatrix}^{-1}
    \begin{bmatrix}
        3  \\
        -3 \\
        1
    \end{bmatrix}
    \\
    \implies
    \begin{bmatrix}
        k_1 \\
        k_2 \\
        k_3
    \end{bmatrix}
     & =
    \begin{bmatrix}
        1 & 1  & 1  \\
        4 & 3  & 2  \\
        0 & -1 & -1
    \end{bmatrix}
    \begin{bmatrix}
        3  \\
        -3 \\
        1
    \end{bmatrix}
    =
    \begin{bmatrix}
        1 \\
        5 \\
        2
    \end{bmatrix}
\end{align*}
Thus, the state feedback gain is given by
\begin{equation*}
    \boxed{
        K
        =
        \begin{bmatrix}
            1 & 5 & 2
        \end{bmatrix}
    }
\end{equation*}

\subsubsection*{(b) Zero-input response of the feedback system}

For the zero-input response of the system, we have
\begin{equation*}
    x[k+1] = (A - BK) x[k]
\end{equation*}

Now, with the value of \( K \) obtained in part (a), we have
\begin{align*}
    A - BK
     & =
    \begin{bmatrix}
        1 & 1 & -2 \\
        0 & 1 & 1  \\
        0 & 0 & 1
    \end{bmatrix}
    -
    \begin{bmatrix}
        1 & 5 & 2 \\
        0 & 0 & 0 \\
        1 & 5 & 2
    \end{bmatrix}
    =
    \begin{bmatrix}
        0  & -4 & -4 \\
        0  & 1  & 1  \\
        -1 & -5 & -1
    \end{bmatrix}
    \\
    \implies
    (A - BK)^2
     & =
    \begin{bmatrix}
        0  & -4 & -4 \\
        0  & 1  & 1  \\
        -1 & -5 & -1
    \end{bmatrix}
    \begin{bmatrix}
        0  & -4 & -4 \\
        0  & 1  & 1  \\
        -1 & -5 & -1
    \end{bmatrix}
    =
    \begin{bmatrix}
        4  & 16 & 0 \\
        -1 & -4 & 0 \\
        1  & 4  & 0
    \end{bmatrix}
    \\
    \implies
    (A - BK)^3
     & =
    \begin{bmatrix}
        4  & 16 & 0 \\
        -1 & -4 & 0 \\
        1  & 4  & 0
    \end{bmatrix}
    \begin{bmatrix}
        0  & -4 & -4 \\
        0  & 1  & 1  \\
        -1 & -5 & -1
    \end{bmatrix}
    =
    \begin{bmatrix}
        0 & 0 & 0 \\
        0 & 0 & 0 \\
        0 & 0 & 0
    \end{bmatrix}
\end{align*}

As we can see from above, \( {(A - BK)}^3 = 0 \), thereby the matrix \( (A - BK) \) is nilpotent of order 3, which gives that \( {(A - BK)}^k = 0 \) for \( k \geq 3 \).

Thereby, the \underline{state variables become identically zero for \( k \geq 3 \)}.
