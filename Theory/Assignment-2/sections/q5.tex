\section*{Problem 5}

Check whether the function is
(i) Locally Lipschitz
(ii) Globally Lipschitz
and mention the Lipschitz Constant:
\( f (x) = -x + 4\sin(x) \)

\subsection*{Solution}

Given \( f(x) = -x + 4\sin(x) \).

\begin{lemma}{Triangle inequality:}
    For any \( x, y \in \mathbb{R} \), we have \( \lvert x + y \rvert \leq \lvert x \rvert + \lvert y \rvert \).
\end{lemma}

\begin{lemma}
    For any \( x, y \in \mathbb{R} \), we have \( \lvert \sin(x) - \sin(y) \rvert \leq \lvert x - y \rvert \).
\end{lemma}
\begin{proof}
    It follows from the Mean Value Theorem.
    Since \( \sin(x) \) is continuous and differentiable on \( \mathbb{R} \), there exists a \( c \) between \( x \) and \( y \) such that
    \[
        \sin'(c) = \cos(c) = \frac{\sin(x) - \sin(y)}{x - y}
    \]
    Since \( \lvert \cos(c) \rvert \leq 1 \) for all \( c \in \mathbb{R} \), we have
    \[
        \lvert \sin(x) - \sin(y) \rvert
        =
        \lvert \cos(c) \rvert \lvert x - y \rvert
        \leq
        \lvert x - y \rvert
    \]
\end{proof}

\subsubsection*{Globally Lipschitz}

To show that \( f(x) \) is globally Lipschitz, we need to show that there exists a constant \( L > 0 \) such that
\[
    \lvert f(x_1) - f(x_2) \rvert \leq L \lvert x_1 - x_2 \rvert
\]
for all \( x_1, x_2 \in \mathbb{R} \).

We can see that
\begin{align*}
    \lvert f(x_1) - f(x_2) \rvert
     & =
    \lvert (-x_1 + 4\sin(x_1)) - (-x_2 + 4\sin(x_2)) \rvert
    \\ & =
    \lvert x_2 - x_1 + 4\sin(x_1) - 4\sin(x_2) \rvert
    \\ & \leq
    \lvert x_2 - x_1 \rvert + \lvert 4\sin(x_1) - 4\sin(x_2) \rvert
    \\ & =
    \lvert x_2 - x_1 \rvert + 4 \lvert \sin(x_1) - \sin(x_2) \rvert
    \\ & \leq
    \lvert x_2 - x_1 \rvert + 4 \lvert x_1 - x_2 \rvert
    \\ & = 5 \lvert x_2 - x_1 \rvert
\end{align*}
where the first inequality follows from the triangle inequality for norms, and the second inequality follows from the lemma.

Therefore, \( f(x) \) \underline{is globally Lipschitz} with Lipschitz constant \( L = 5 \).

\subsubsection*{Locally Lipschitz}

To show that \( f(x) \) is locally Lipschitz, we need to show that for every \( x \in \mathbb{R} \), there exists a neighborhood \( U \) of \( x \) such that \( f(x) \) is Lipschitz on \( U \).

Since the function is globally Lipschitz, it \underline{is locally Lipschitz} as well.~\qed
