\section*{Problem 6}

Suppose, \( f_1 : \mathbb{R} \rightarrow \mathbb{R} \) and \( f_2 : \mathbb{R} \rightarrow \mathbb{R} \) are two locally Lipschitz functions.
Show that, \( f_1 + f_2 \) and \( f_2 \circ f_1 \) are locally Lipschitz.

\subsection*{Solution}

We know that \( \mathbb{R} \) is a normed vector space with the Euclidean norm \( \lVert \cdot \rVert \).

Given \( f_1 : \mathbb{R} \rightarrow \mathbb{R} \) and \( f_2 : \mathbb{R} \rightarrow \mathbb{R} \) are two locally Lipschitz functions, we have
\begin{align*}
    \exists \, L_1 > 0 \, \text{such that} \, \forall \, x_1, x_2 \in \mathbb{R}, \, \lVert f_1(x_1) - f_1(x_2) \rVert \leq L_1 \lVert x_1 - x_2 \rVert, \\
    \exists \, L_2 > 0 \, \text{such that} \, \forall \, y_1, y_2 \in \mathbb{R}, \, \lVert f_2(y_1) - f_2(y_2) \rVert \leq L_2 \lVert y_1 - y_2 \rVert .
\end{align*}
where \( L_1 \) and \( L_2 \) are the Lipschitz constants for \( f_1 \) and \( f_2 \) respectively.

\begin{lemma}{Triangle inequality for norms:}
    For any \( x, y \in \mathbb{R} \), we have \( \lVert x + y \rVert \leq \lVert x \rVert + \lVert y \rVert \).
\end{lemma}

\subsubsection*{\( f_1 + f_2 \) is locally Lipschitz}

Let \( f_3 = f_1 + f_2 \).
We need to show that \( f_3 \) is locally Lipschitz.

Consider \( x_1, x_2 \in \mathbb{R} \).
We have
\begin{align*}
    \lVert f_3(x_1) - f_3(x_2) \rVert
     & =
    \lVert (f_1 + f_2)(x_1) - (f_1 + f_2)(x_2) \rVert
    \\ & =
    \lVert f_1(x_1) + f_2(x_1) - f_1(x_2) - f_2(x_2) \rVert
    \\ & =
    \lVert f_1(x_1) - f_1(x_2) + f_2(x_1) - f_2(x_2) \rVert
    \\ & \leq
    \lVert f_1(x_1) - f_1(x_2) \rVert + \lVert f_2(x_1) - f_2(x_2) \rVert
    \\ & \leq
    L_1 \lVert x_1 - x_2 \rVert + L_2 \lVert x_1 - x_2 \rVert
    \\ & = (L_1 + L_2) \lVert x_1 - x_2 \rVert
\end{align*}
where the first inequality follows from the triangle inequality for norms.
The \( L_1 \) and \( L_2 \) Lipschitz constants here are for the local regions.

Therefore, \( f_3 = \) \underline{\( f_1 + f_2 \) is locally Lipschitz} with Lipschitz constant \( L_3 = L_1 + L_2 \).

\subsubsection*{\( f_2 \circ f_1 \) is locally Lipschitz}

Let \( f_4 = f_2 \circ f_1 \).
We need to show that \( f_4 \) is locally Lipschitz.

Consider \( x_1, x_2 \in \mathbb{R} \).
We have
\begin{align*}
    \lVert f_4(x_1) - f_4(x_2) \rVert
     & =
    \lVert f_2(f_1(x_1)) - f_2(f_1(x_2)) \rVert
    \\ & \leq
    L_2 \lVert f_1(x_1) - f_1(x_2) \rVert
    \\ & \leq
    L_2 L_1 \lVert x_1 - x_2 \rVert
\end{align*}

Therefore, \( f_4 = \) \underline{\( f_2 \circ f_1 \) is locally Lipschitz} with Lipschitz constant \( L_4 = L_1 L_2 \).\qed
