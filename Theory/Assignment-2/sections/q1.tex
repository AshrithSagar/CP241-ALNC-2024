\section*{Problem 1}

Compute the mathematical expression for the solution of the following system:
\begin{equation*}
    \dot x(t)
    =
    \begin{bmatrix}
        \frac{-t}{1+t^2} & 1                 \\
        0                & \frac{-4t}{1+t^2}
    \end{bmatrix}
    x(t)
\end{equation*}
Assume initial condition \( x(0) = x_0 \).
What happens if \( t \rightarrow \infty \)?
Explain your answer.

\subsection*{Solution}

Given the homogeneous linear time-varying system:
\begin{equation*}
    \dot{x}(t)
    =
    A(t) x(t),
    \quad
    A(t)
    =
    \begin{bmatrix}
        \frac{-t}{1+t^2} & 1                 \\
        0                & \frac{-4t}{1+t^2}
    \end{bmatrix},
    \quad
    x(0) = x_0
\end{equation*}
Define \( M(t) \) as follows:
\begin{equation*}
    M(t)
    \triangleq
    \int_{0}^{t} A(l) \, dl
\end{equation*}
Provided \( M(t) \) and \( A(t) \) commute, we have the state transition matrix as
\begin{align*}
    \Phi(t, 0)
     & =
    e^{M(t)} x_0
    \\
    \implies
    e^{M(t)}
     & =
    I + M(t) + \frac{1}{2!} M^2(t) + \frac{1}{3!} M^3(t) + \ldots
\end{align*}
Computing the first few powers of the matrix \( M(t) \), we have
\begin{align*}
    \implies
    M(t)
        & =
    \int_{0}^{t}
    \begin{bmatrix}
        \frac{-l}{1+l^2} & 1                 \\
        0                & \frac{-4l}{1+l^2}
    \end{bmatrix}
    \, dl
    =
    \begin{bmatrix}
        \int_{0}^{t} \frac{-l}{1+l^2} \, dl
        &
        \int_{0}^{t} 1 \, dl
        \\
        0
        &
        \int_{0}^{t} \frac{-4l}{1+l^2} \, dl
    \end{bmatrix}
    \\ & =
    \begin{bmatrix}
        -\frac{1}{2} \ln(1+t^2)
        &
        t
        \\
        0
        &
        -2 \ln(1+t^2)
    \end{bmatrix}
    \\
    \implies
    M^2(t)
    & =
    M(t) \, M(t)
    =
    \begin{bmatrix}
        \frac{1}{4} \ln^2(1+t^2)
        & &
        -\frac{5}{2} t \ln(1+t^2)
        \\ \\
        0
        & &
        4 \ln^2(1+t^2)
    \end{bmatrix}
    \\
    \implies
    M^3(t)
    & =
    M(t) \, M^2(t)
    =
    \begin{bmatrix}
        -\frac{1}{8} \ln^3(1+t^2)
        & &
        \frac{21}{4} t \ln^2(1+t^2)
        \\ \\
        0
        & &
        -8 \ln^3(1+t^2)
    \end{bmatrix}
    \\
    \implies
    M^4(t)
    & =
    M(t) \, M^3(t)
    =
    \begin{bmatrix}
        \frac{1}{16} \ln^4(1+t^2)
        & &
        -\frac{85}{8} t \ln^3(1+t^2)
        \\ \\
        0
        & &
        16 \ln^4(1+t^2)
    \end{bmatrix}
\end{align*}

By inspection, we can see that \( M^{k} (t) \) for \( k = 1, 2, 3, \ldots \) is given by,
\begin{equation*}
    M^{k} (t)
    =
    \begin{bmatrix}
        \displaystyle
        {(-1)}^{k} \frac{1}{2^k} \ln^{k}(1+t^2)
        & & &
        \displaystyle
        {(-1)}^{k-1} \left( \frac{4^k-1}{3} \right) \frac{1}{2^{k-1}} t \ln^{k-1}(1+t^2)
        \\ \\
        0
        & & &
        \displaystyle
        {(-1)}^{k} 2^{k} \ln^{k}(1+t^2)
    \end{bmatrix}
\end{equation*}

Now, we can compute \( e^{M(t)} \) as follows.
Let
\begin{equation*}
    e^{M(t)}
      =
    \begin{bmatrix}
        a(t) & b(t) \\
        0    & c(t)
    \end{bmatrix}
\end{equation*}
where we have the elements of the matrix as
\begin{align*}
    a(t)
    & =
    1 + \sum_{k=1}^{\infty} \frac{1}{k!} {(-1)}^{k} \frac{1}{2^k} \ln^{k}(1+t^2)
    =
    \sum_{k=0}^{\infty} \frac{1}{k!} {\left( \frac{-1}{2} \ln(1+t^2) \right)}^{k}
    =
    e^{\frac{-1}{2} \ln(1+t^2)}
    \\
    c(t)
    & =
    1 + \sum_{k=1}^{\infty} \frac{1}{k!} {(-1)}^{k} 2^{k} \ln^{k}(1+t^2)
    =
    \sum_{k=0}^{\infty} \frac{1}{k!} {(-2 \ln(1+t^2))}^{k}
    =
    e^{-2 \ln(1+t^2)}
    \\
    b(t)
    & =
    \sum_{k=1}^{\infty} \frac{1}{k!} {(-1)}^{k-1} \left( \frac{4^k-1}{3} \right) \frac{1}{2^{k-1}} t \ln^{k-1}(1+t^2)
    \\ & =
    \frac{-2t}{3\ln(1+t^2)} \sum_{k=1}^{\infty} \frac{1}{k!} {\left( \frac{1}{2} \ln(1+t^2) \right)}^{k} (4^k - 1)
    \\ & =
    \frac{-2t}{3\ln(1+t^2)} \left(
        \sum_{k=1}^{\infty} \frac{1}{k!} {\left( \frac{1}{2} \ln(1+t^2) \right)}^{k} 4^k
        -
        \sum_{k=1}^{\infty} \frac{1}{k!} {\left( \frac{1}{2} \ln(1+t^2) \right)}^{k}
     \right)
     \\ & =
    \frac{-2t}{3\ln(1+t^2)} \left(
        e^{2 \ln(1+t^2)}
        -
        e^{\frac{1}{2} \ln(1+t^2)}
     \right)
     \\ & =
    \frac{-2t}{3\ln(1+t^2)} \left(
        {(1+t^2)}^{2}
        -
        {(1+t^2)}^{\frac{1}{2}}
     \right)
\end{align*}

Thereby, the state transition matrix is given by
\begin{equation*}
    \Phi(t, 0)
    =
    \begin{bmatrix}
        e^{\frac{-1}{2} \ln(1+t^2)}
        &
        \frac{-2t}{3\ln(1+t^2)} \left( {(1+t^2)}^{2} - {(1+t^2)}^{\frac{1}{2}} \right)
        \\
        0
        &
        e^{-2 \ln(1+t^2)}
    \end{bmatrix}
\end{equation*}
and the solution is given by
\begin{equation*}
    \phi(t)
    =
    \Phi(t, 0) x_0
    =
    \boxed{
    \begin{bmatrix}
        e^{\frac{-1}{2} \ln(1+t^2)}
        &
        \frac{-2t}{3\ln(1+t^2)} \left( {(1+t^2)}^{2} - {(1+t^2)}^{\frac{1}{2}} \right)
        \\
        0
        &
        e^{-2 \ln(1+t^2)}
    \end{bmatrix}
    x_0
    }
\end{equation*}
