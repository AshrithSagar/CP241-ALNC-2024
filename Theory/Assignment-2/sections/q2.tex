\section*{Problem 2}

Verify that, \( \phi_1(t) = [1/t^2, -1/t]^T \) and \( \phi_2(t) = [2/t^3, -1/t^2]^T \) are two solutions of the system \( \dot x(t) = A(t)x(t) \) with
\begin{equation*}
    A(t)
    =
    \begin{bmatrix}
        \frac{-4}{t} & \frac{-2}{t^2} \\
        1            & 0
    \end{bmatrix}
\end{equation*}

\begin{enumerate}[label= (\alph*)]
    \item Determine the state transition matrix \( \Phi(t,\tau) \) for the system.
    \item Determine a solution \( \phi \) for the system that satisfies the initial condition \( x(1) = {[1, 1]}^T \).
\end{enumerate}

\subsection*{Solution}

We can verify that \( \phi_1(t) \) and \( \phi_2(t) \) are solutions of the system as follows:

\begin{align*}
    \phi_1(t)
     & =
    \begin{bmatrix}
        1/t^2 \\
        -1/t
    \end{bmatrix}
    \implies
    \dot \phi_1(t)
    =
    \begin{bmatrix}
        -2/t^3 \\
        1/t^2
    \end{bmatrix}
    \\
    \implies
    A(t)\phi_1(t)
     & =
    \begin{bmatrix}
        \frac{-4}{t} & \frac{-2}{t^2} \\
        1            & 0
    \end{bmatrix}
    \begin{bmatrix}
        1/t^2 \\
        -1/t
    \end{bmatrix}
    =
    \begin{bmatrix}
        -2/t^3 \\
        1/t^2
    \end{bmatrix}
    =
    \dot \phi_1(t)
\end{align*}
\begin{align*}
    \phi_2(t)
     & =
    \begin{bmatrix}
        2/t^3 \\
        -1/t^2
    \end{bmatrix}
    \implies
    \dot \phi_2(t)
    =
    \begin{bmatrix}
        -6/t^4 \\
        2/t^3
    \end{bmatrix}
    \\
    \implies
    A(t)\phi_2(t)
     & =
    \begin{bmatrix}
        \frac{-4}{t} & \frac{-2}{t^2} \\
        1            & 0
    \end{bmatrix}
    \begin{bmatrix}
        2/t^3 \\
        -1/t^2
    \end{bmatrix}
    =
    \begin{bmatrix}
        -6/t^4 \\
        2/t^3
    \end{bmatrix}
    =
    \dot \phi_2(t)
\end{align*}

Hence, \underline{\( \phi_1(t) \) and \( \phi_2(t) \) are solutions of the system}.

\subsubsection*{(a) State transition matrix \( \Phi(t,\tau) \)}

We can verify that the solutions \( \phi_1(t) \) and \( \phi_2(t) \) are linearly independent.
\begin{equation*}
    \det \begin{bmatrix}
        \phi_1(t) & \phi_2(t)
    \end{bmatrix}
    =
    \det \begin{bmatrix}
        1/t^2 & 2/t^3  \\
        -1/t  & -1/t^2
    \end{bmatrix}
    =
    - \frac{1}{t^4} + \frac{2}{t^4}
    =
    \frac{1}{t^4}
    \neq 0
    \; \forall t \in \mathbb{R}/\{0\}
\end{equation*}

Thereby, the state transition matrix \( \Phi(t,\tau) \) is given by
\begin{equation*}
    \Phi(t,\tau)
    =
    \begin{bmatrix}
        \phi_1(t) & \phi_2(t)
    \end{bmatrix}
    \begin{bmatrix}
        \phi_1(\tau) & \phi_2(\tau)
    \end{bmatrix}^{-1}
\end{equation*}
where \( U(t) = \begin{bmatrix} \phi_1(t) & \phi_2(t) \end{bmatrix} \) constitutes the fundamental solution matrix.
\begin{equation*}
    \implies
    \begin{bmatrix}
        \phi_1(\tau) & \phi_2(\tau)
    \end{bmatrix}^{-1}
    =
    \tau^4
    \begin{bmatrix}
        -1/\tau^2 & -2/\tau^3 \\
        1/\tau    & 1/\tau^2
    \end{bmatrix}
    =
    \begin{bmatrix}
        -\tau^2 & -2\tau \\
        \tau^3  & \tau^2
    \end{bmatrix}
\end{equation*}
\begin{equation*}
    \implies
    \Phi(t,\tau)
    =
    \begin{bmatrix}
        1/t^2 & 2/t^3  \\
        -1/t  & -1/t^2
    \end{bmatrix}
    \begin{bmatrix}
        -\tau^2 & -2\tau \\
        \tau^3  & \tau^2
    \end{bmatrix}
\end{equation*}
\begin{equation*}
    \therefore
    \boxed{
        \Phi(t,\tau)
        =
        \begin{bmatrix}
            \left( -\cfrac{\tau^2}{t^2} + \cfrac{2\tau^3}{t^3} \right)
             &  &
            \left( -\cfrac{2\tau}{t^2} + \cfrac{2\tau^2}{t^3}  \right)
            \\ \\
            \left( \cfrac{\tau^2}{t} - \cfrac{\tau^3}{t^2}  \right)
             &  &
            \left( \cfrac{2\tau}{t} - \cfrac{\tau^2}{t^2}  \right)
        \end{bmatrix}
    }
\end{equation*}

\subsubsection*{(b) Solution satisfying the initial condition \( x(1) = {[1, 1]}^T \)}

We know that the solution is given by
\begin{equation*}
    x(t)
    =
    \Phi(t,\tau)x(\tau)
\end{equation*}
Given the initial condition \( x(1) = {[1, 1]}^T \), we have that \( \tau = 1 \) and \( x(1) = {[1, 1]}^T \), thereby
\begin{align*}
    \implies
    x(t)
     & =
    \Phi(t,1)
    \begin{bmatrix}
        1 \\
        1
    \end{bmatrix}
    \\ & =
    \begin{bmatrix}
        \left( -\cfrac{1}{t^2} + \cfrac{2}{t^3} \right)
         &  &
        \left( -\cfrac{2}{t^2} + \cfrac{2}{t^3} \right)
        \\ \\
        \left( \cfrac{1}{t} - \cfrac{1}{t^2} \right)
         &  &
        \left( \cfrac{2}{t} - \cfrac{1}{t^2} \right)
    \end{bmatrix}
    \begin{bmatrix}
        1 \\
        1
    \end{bmatrix}
    \\ & =
    \begin{bmatrix}
        \left( -\cfrac{1}{t^2} + \cfrac{2}{t^3} -\cfrac{2}{t^2} + \cfrac{2}{t^3} \right)
        \\ \\
        \left( \cfrac{1}{t} - \cfrac{1}{t^2} + \cfrac{2}{t} - \cfrac{1}{t^2} \right)
    \end{bmatrix}
\end{align*}
\begin{equation*}
    \therefore
    \boxed{
        x(t)
        =
        \begin{bmatrix}
            \left( -\cfrac{3}{t^2} + \cfrac{4}{t^3} \right)
            \\ \\
            \left( \cfrac{3}{t} - \cfrac{2}{t^2} \right)
        \end{bmatrix}
    }
\end{equation*}
