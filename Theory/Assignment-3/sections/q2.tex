\section*{Problem 2}

Consider the system:
\begin{equation*}
    \begin{gathered}
        \dot{x}_{1}=-x_{1}+a x_{2}-b x_{1} x_{2}+x_{2}^{2} \\
        \dot{x}_{2}=-(a+b) x_{1}+b x_{1}^{2}-x_{1} x_{2}
    \end{gathered}
\end{equation*}
where \( a>0, b \neq 0 \).

\begin{enumerate}[label= (\alph*)]
    \item Find all the equilibrium points of the system.
    \item Determine the type of each equilibrium point for all values of \( a>0, b \neq 0 \).
    \item Construct the phase portrait for the following cases and discuss the qualitative behavior of the system:
          \begin{enumerate}[label= (\roman*)]
              \item \( \mathrm{a}=\mathrm{b}=1 \)
              \item \( \mathrm{a}=1, \mathrm{~b}=-\frac{1}{2} \)
          \end{enumerate}
\end{enumerate}

\subsection*{Solution}

\subsubsection*{(a) Equilibrium Points}

The equilibrium points are obtained by setting state derivatives to zero.
Starting with the second equation, we get
\begin{align*}
    \implies
    \dot x_2
     & =
    0
    =
    -(a + b) x_1 + b x_1^2 - x_1 x_2
    =
    x_1 \left( b x_1 - x_2 - (a + b) \right)
    =
    0
    \\
    \implies
    x_1
     & =
    0
    \quad \text{or} \quad
    b x_1 - x_2 - (a + b)
    =
    0
\end{align*}
Putting \( x_1 = 0 \) in the first equation, we get
\begin{align*}
    \implies
    \dot x_1
     & =
    0
    =
    -x_1 + a x_2 - b x_1 x_2 + x_2^2
    =
    a x_2 + x_2^2
    =
    x_2 \left( a + x_2 \right)
    =
    0
    \\
    \implies
    x_2
     & =
    0
    \quad \text{or} \quad
    a + x_2
    =
    0
\end{align*}
For the other equilibrium points, we can see that
\begin{align*}
    b x_1 - x_2 - (a + b)
     & =
    0
    \\
    \implies
    x_2
     & =
    b x_1 - (a + b)
\end{align*}
Substituting this in the first equation, we get
\begin{align*}
    \implies
    \dot x_1
     & =
    0
    =
    -x_1 + a \left( b x_1 - (a + b) \right) - b x_1 \left( b x_1 - (a + b) \right) + \left( b x_1 - (a + b) \right)^2
    \\
    \implies
    0
     & =
    -x_1 + a b x_1 - a (a + b) - b x_1 \left( b x_1 - (a + b) \right) + \left( b x_1 - (a + b) \right)^2
    \\
    \implies
    0
     & =
    -x_1 + a b x_1 - a (a + b) - \cancel{b^2 x_1^2} + b (a + b) x_1 + \cancel{b^2 x_1^2} - 2 b (a + b) x_1 + (a + b)^2
    \\
    \implies
    0
     & =
    -x_1 + a b x_1 - b (a + b) x_1 + b(a + b)
\end{align*}
\begin{align*}
    \implies
    0
     & =
    x_1 \left( -1 + a b - b (a + b) \right) + b(a + b)
    \\
    \implies
    0
     & =
    x_1 \left( -1 + \cancel{a b} - \cancel{b a} - b^2 \right) + b(a + b)
    \\
    \implies
    x_1
     & =
    \frac{b(a + b)}{b^2 + 1}
\end{align*}
Putting this back in the equation for \( x_2 \), we get
\begin{align*}
    x_2
     & =
    b x_1 - (a + b)
    =
    b \left( \frac{b(a + b)}{b^2 + 1} \right) - (a + b)
    \\ & =
    \frac{b^2(a + b) - (a + b)(b^2 + 1)}{b^2 + 1}
    =
    \frac{\cancel{b^2 a} + \cancel{b^3} - \cancel{a b^2} - a - \cancel{b^3} - b}{b^2 + 1}
    \\
    \implies
    x_1
     & =
    \frac{-(a+b)}{b^2 + 1}
\end{align*}

Thereby, the equilibrium points are
\begin{equation*}
    \boxed{
        x^* =
        \left \{ \;
        \begin{bmatrix}
            0 \\
            0
        \end{bmatrix}, \; \;
        \begin{bmatrix}
            0 \\
            -a
        \end{bmatrix}, \; \;
        \frac{a+b}{b^2 + 1}
        \begin{bmatrix}
            -1 \\
            b
        \end{bmatrix} \;
        \right \}
    }
\end{equation*}

\subsubsection*{(b) Type of Equilibrium Points}

We can calulate the Jacobian matrix of the system by evaluating
\begin{align*}
    \frac{\partial f_1}{\partial x_1}
     & =
    -1 - b x_2
     &
    \frac{\partial f_1}{\partial x_2}
     & =
    a - b x_1 + 2 x_2
    \\
    \frac{\partial f_2}{\partial x_1}
     & =
    -(a + b) + 2 b x_1 - x_2
     &
    \frac{\partial f_2}{\partial x_2}
     & =
    -x_1
\end{align*}
giving the Jacobian matrix as
\begin{equation*}
    J(x) =
    \begin{bmatrix}
        -1 - b x_2               & a - b x_1 + 2 x_2 \\
        -(a + b) + 2 b x_1 - x_2 & -x_1
    \end{bmatrix}
\end{equation*}
Around the equilibrium point \( x^* = \begin{bmatrix} 0 \\ 0 \end{bmatrix} \implies
\begin{bmatrix} \delta \dot x_1 \\ \delta \dot x_2 \end{bmatrix}
=
\begin{bmatrix}
    -1     & a \\
    -(a+b) & 0
\end{bmatrix}
\begin{bmatrix} \delta x_1 \\ \delta x_2 \end{bmatrix}\).
The characteristic polynomial is
\begin{align*}
    \text{det}
    \begin{bmatrix}
        -1 - \lambda & a        \\
        -(a + b)     & -\lambda
    \end{bmatrix}
    =
    (-1 - \lambda)(-\lambda) - a(a + b)
    =
    \lambda^2 + \lambda + a(a + b)
\end{align*}
The eigenvalues are threreby \( \lambda = \cfrac{-1 \pm \sqrt{1 - 4a(a + b)}}{2} \).
For real part \(Re(\lambda)< 0\), it is a stable equilibrium point.
For \( 1 - 4a(a + b) < 0 \), the eigenvalues are complex and the equilibrium point is focus.
For \( 1 - 4a(a + b) \geq 0 \), the equilibrium point is a node.

For the equilibrium point \( x^* = \begin{bmatrix} 0 \\ -a \end{bmatrix} \implies
\begin{bmatrix} \delta \dot x_1 \\ \delta \dot x_2 \end{bmatrix}
=
\begin{bmatrix}
    ab-1 & -a \\
    -b   & 0
\end{bmatrix}
\begin{bmatrix} \delta x_1 \\ \delta x_2 \end{bmatrix}\).\\
The characteristic polynomial is
\begin{align*}
    \text{det}
    \begin{bmatrix}
        ab - 1 - \lambda & -a       \\
        -b               & -\lambda
    \end{bmatrix}
    =
    (ab - 1 - \lambda)(-\lambda) - a b
    =
    \lambda^2 + \lambda (1 - ab) - ab
\end{align*}
The eigenvalues are threreby \( \lambda = ab, -1 \).
If \( b > 0 \) implies a saddle point, and \( b < 0 \) implies a stable point.

For the equilibrium point \( x^* = \cfrac{a+b}{b^2 + 1} \begin{bmatrix} -1 \\ b \end{bmatrix} \)
\[ \implies
    \begin{bmatrix} \delta \dot x_1 \\ \delta \dot x_2 \end{bmatrix}
    =
    \begin{bmatrix}
        -1 + \frac{b(a+b)}{b^2+1}                              & a - \frac{b^2(a+b)}{b^2+1}-\frac{2(a+b)}{b^+1} \\
        -a - b + \frac{2b^2(a+b)}{b^2+1} + \frac{(a+b)}{b^2+1} & -\frac{b(a+b)}{b^2+1}
    \end{bmatrix}
    \begin{bmatrix} \delta x_1 \\ \delta x_2 \end{bmatrix}
\]

\clearpage
\subsubsection*{(c) Phase Portrait}

\begin{enumerate}[label= (\roman*)]
    \item For \( a = b = 1 \), the equilibrium points are
          \begin{align*}
              x^* =
              \left \{ \;
              \begin{bmatrix}
                  0 \\
                  0
              \end{bmatrix}, \; \;
              \begin{bmatrix}
                  0 \\
                  -1
              \end{bmatrix}, \; \;
              \begin{bmatrix}
                  -1 \\
                  1
              \end{bmatrix} \;
              \right \}
          \end{align*}
          The phase portrait is shown in Figure~\ref{fig:q2-1}.
          \begin{figure}[!ht]
              \centering
              \includegraphics[width=0.85\textwidth]{figures/q2/q2-1.png}
              \vspace*{-2.5em}
              \caption{
                  Phase Portrait for \( a = b = 1 \)
              }\label{fig:q2-1}
          \end{figure}

    \item For \( a = 1, b = -\cfrac{1}{2} \), the equilibrium points are
          \begin{align*}
              x^* =
              \left \{ \;
              \begin{bmatrix}
                  0 \\
                  0
              \end{bmatrix}, \; \;
              \begin{bmatrix}
                  0 \\
                  -1
              \end{bmatrix}, \; \;
              \begin{bmatrix}
                  -2/5 \\
                  -1/5
              \end{bmatrix} \;
              \right \}
          \end{align*}
          The phase portrait is shown in Figure~\ref{fig:q2-2}.
          \begin{figure}[!ht]
              \centering
              \includegraphics[width=0.85\textwidth]{figures/q2/q2-2.png}
              \vspace*{-2.5em}
              \caption{
                  Phase Portrait for \( a = 1, b = -\cfrac{1}{2} \)
              }\label{fig:q2-2}
          \end{figure}
\end{enumerate}
