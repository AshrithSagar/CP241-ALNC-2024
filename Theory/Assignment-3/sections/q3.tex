\section*{Problem 3}

\begin{enumerate}[label= (\alph*)]
      \item Prove that, \( \frac{d}{d t}\left(e^{A t}\right)=A e^{A t}=e^{A t} A \), where \( A \) is a constant state-matrix for the system \( \dot{x}=A x \).
      \item Prove that, \( X(t)=e^{A t} C e^{F t} \) is a solution to the matrix equation \( \dot{X}=A X+X F \), where \( X(0)=C \).
            Assume, \( A, C, F \) are constant matrices.
      \item An LTV System is given by \( \dot{x}(t)=A(t) x(t) \). Let, \( P(t) \) be a transformation matrix such that \( z=P(t) x \).
            If the transformation renders the system to an LTI system \( \dot{z}=F z \), then show that \( P(t)=e^{F t} P(0) e^{-M(t)} \), where \( M(t)=\int_{\tau=0}^{t} A(\tau) d \tau \).
            Note that, \( F \) is a constant matrix.
            Assume, \( M(t) \) and \( A(t) \) commute.
\end{enumerate}

\subsection*{Solution}

\subsubsection*{(a) \( \frac{d}{d t}\left(e^{A t}\right)=A e^{A t}=e^{A t} A \)}

\begin{proof}
      We know that the matrix exponential is defined as:
      \begin{equation*}
            e^{A t}
            =
            \sum_{k=0}^{\infty} \frac{{(A t)}^{k}}{k !}
            =
            I + \sum_{k=1}^{\infty} \frac{{(A t)}^{k}}{k !}
      \end{equation*}
      Differentiating with respect to \( t \), we get:
      \begin{align*}
            \frac{d}{d t}\left(e^{A t}\right)
             & =
            \frac{d}{d t}\left(I + \sum_{k=1}^{\infty} \frac{{(A t)}^{k}}{k !}\right)
            =
            \sum_{k=1}^{\infty} \frac{d}{d t}\left(\frac{{(A t)}^{k}}{k !}\right)
            =
            \sum_{k=1}^{\infty} A^k \frac{d}{d t}\left(\frac{t^k}{k !}\right)
            =
            \sum_{k=1}^{\infty} A^k \frac{t^{k-1}}{(k-1) !}
      \end{align*}
      Now, from here, we can see that
      \begin{align*}
            \frac{d}{d t}\left(e^{A t}\right)
             & =
            \sum_{k=1}^{\infty} A^k \frac{t^{k-1}}{(k-1) !}
            =
            A \left( \sum_{k=1}^{\infty} \frac{{(A t)}^{k-1}}{(k-1) !} \right)
            =
            A \left( \sum_{k=0}^{\infty} \frac{{(A t)}^{k}}{k !} \right)
            = A e^{A t}
            \\
            \frac{d}{d t}\left(e^{A t}\right)
             & =
            \sum_{k=1}^{\infty} A^k \frac{t^{k-1}}{(k-1) !}
            =
            \sum_{k=0}^{\infty} \frac{{(A t)}^{k}}{k !} A
            =
            \left( \sum_{k=0}^{\infty} \frac{{(A t)}^{k}}{k !} \right) A
            = e^{A t} A
      \end{align*}
\end{proof}

\subsubsection*{(b) \( X(t)=e^{A t} C e^{F t} \) is a solution to \( \dot{X}=A X+X F \)}

\begin{proof}
      We are given that \( X(t)=e^{A t} C e^{F t} \).
      Differentiating \( X(t) \) with respect to \( t \), we get:
      \begin{align*}
            \dot{X}(t)
             & =
            \frac{d}{d t}\left(e^{A t} C e^{F t}\right)
            =
            \frac{d}{d t}\left(e^{A t}\right) C e^{F t} + e^{A t} C \frac{d}{d t}\left(e^{F t}\right)
            \\ & =
            A e^{A t} C e^{F t} + e^{A t} C e^{F t} F
            =
            A X(t) + X(t) F
      \end{align*}
      \( \therefore X(t) \) is a solution to the matrix equation \( \dot{X}=A X+X F \), with constant matrices \( A, C, F \).
\end{proof}

\subsubsection*{(c) \( P(t)=e^{F t} P(0) e^{-M(t)} \)}

\begin{proof}
      Given the LTV system \( \dot{x}(t)=A(t) x(t) \), we have a transformation matrix \( z=P(t) x \) such that \( \dot{z}=F z \), where \( F \) is a constant matrix, which converts it to an LTI system.

      The solution to the LTV system is given by, since \( M(t) \) and \( A(t) \) commute,
      \begin{equation*}
            \dot{x}(t)
            =
            A(t) x(t)
            \implies
            x(t)
            =
            e^{M(t)} x(0),
            \quad
            M(t)
            =
            \int_{\tau=0}^{t} A(\tau) d \tau
      \end{equation*}
      The solution to the LTI system is given by,
      \begin{equation*}
            \dot{z}(t)
            =
            F z(t)
            \implies
            z(t)
            =
            e^{F t} z(0)
      \end{equation*}

      Now, we can see that, since \( M(t) \) and \( (-M(t)) \) commute:
      \begin{align*}
            e^{-M(t)} x(t)
             & =
            e^{-M(t)} e^{M(t)} x(0)
            =
            e^{-M(t) + M(t)} x(0)
            =
            I x(0)
            =
            x(0)
            \\
            \implies
            P(0) e^{-M(t)} x(t)
             & =
            P(0) x(0)
            =
            z(0)
            \\
            \implies
            e^{F t} P(0) e^{-M(t)} x(t)
             & =
            e^{F t} z(0)
            =
            z(t)
            =
            P(t) x(t)
      \end{align*}
      which is true for all \( t \).
      Thereby, we get
      \begin{equation*}
            \boxed{
                  P(t)
                  =
                  e^{F t} P(0) e^{-M(t)}
            }
      \end{equation*}
\end{proof}
