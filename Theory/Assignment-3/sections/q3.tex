\section*{Problem 3}

\begin{enumerate}[label= (\alph*)]
      \item Prove that, \( \frac{d}{d t}\left(e^{A t}\right)=A e^{A t}=e^{A t} A \), where \( A \) is a constant state-matrix for the system \( \dot{x}=A x \).
      \item Prove that, \( X(t)=e^{A t} C e^{F t} \) is a solution to the matrix equation \( \dot{X}=A X+X F \), where \( X(0)=C \).
            Assume, \( A, C, F \) are constant matrices.
      \item An LTV System is given by \( \dot{x}(t)=A(t) x(t) \). Let, \( P(t) \) be a transformation matrix such that \( z=P(t) x \).
            If the transformation renders the system to an LTI system \( \dot{z}=F z \), then show that \( P(t)=e^{F t} P(0) e^{-M(t)} \), where \( M(t)=\int_{\tau=0}^{t} A(\tau) d \tau \).
            Note that, \( F \) is a constant matrix.
            Assume, \( M(t) \) and \( A(t) \) commute.
\end{enumerate}

\subsection*{Solution}

\subsubsection*{(a) \( \frac{d}{d t}\left(e^{A t}\right)=A e^{A t}=e^{A t} A \)}

\begin{proof}
      We know that the matrix exponential is defined as:
      \begin{equation*}
            e^{A t}=\sum_{k=0}^{\infty} \frac{(A t)^{k}}{k !}
      \end{equation*}
      Differentiating \( e^{A t} \) with respect to \( t \), we get:
      \begin{align*}
            \frac{d}{d t}\left(e^{A t}\right)
             & =
            \frac{d}{d t}\left(\sum_{k=0}^{\infty} \frac{(A t)^{k}}{k !}\right)
            =
            \frac{d}{d t}\left(I + \sum_{k=1}^{\infty} \frac{(A t)^{k}}{k !}\right)
            \\ & =
            \sum_{k=1}^{\infty} \frac{d}{d t}\left(\frac{(A t)^{k}}{k !}\right)
            =
            \sum_{k=1}^{\infty} \frac{{(A t)}^{k-1}}{(k-1) !}
            =
            \sum_{k=0}^{\infty} \frac{{(A t)}^k}{k !}
            =
            A e^{A t}
      \end{align*}
\end{proof}
