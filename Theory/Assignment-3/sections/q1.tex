\section*{Problem 1}

Consider a system is modeled by the following differential equation:
\[
    \ddot y - 2t\dot y - (2 - t^2)y = 0
\]
\begin{enumerate}[label= (\alph*)]
    \item Show that the following are two equivalent state-space representations of the above system:
          \begin{align*}
              \dot x
               & =
              \begin{bmatrix}
                  0       & 1  \\
                  2 - t^2 & 2t
              \end{bmatrix}
              x
              :=
              A_1(t) x
              \\
              \dot z
               & =
              \begin{bmatrix}
                  t & 1 \\
                  1 & t
              \end{bmatrix}
              z
              :=
              A_2(t) z
          \end{align*}
          \textit{Hint:} \(x\) and \(z\) share a common component.
    \item Using the easiest choice, compute the state transition matrix \(\Phi(t, 0)\)
    \item Find \(P(t)\) such that \( z(t) = P(t)x(t) \).
          What property should \(P(t)\) satisfy to be a meaningful transformation matrix?
          Is it holding in this case?
    \item Let, \( \dot{x}(t)=A(t) x+B(t) u \) be a LTV system, and \( z(t)=P(t) x(t) \) is a transformation with \( P(t) \geq 0, \forall t \geq t_{0} \).
          Derive the LTV system that describes the dynamics of \( z \).
    \item Verify your relations in part (d) using \( A_{1}(t), A_{2}(t), P(t) \) of the above example.
\end{enumerate}

\subsection*{Solution}
