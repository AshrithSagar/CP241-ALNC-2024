\section*{Problem 1}

Consider a system is modeled by the following differential equation:
\[
    \ddot y - 2t\dot y - (2 - t^2)y = 0
\]
\begin{enumerate}[label= (\alph*)]
    \item Show that the following are two equivalent state-space representations of the above system:
          \begin{align*}
              \dot x
               & =
              \begin{bmatrix}
                  0       & 1  \\
                  2 - t^2 & 2t
              \end{bmatrix}
              x
              :=
              A_1(t) x
              \\
              \dot z
               & =
              \begin{bmatrix}
                  t & 1 \\
                  1 & t
              \end{bmatrix}
              z
              :=
              A_2(t) z
          \end{align*}
          \textit{Hint:} \(x\) and \(z\) share a common component.
    \item Using the easiest choice, compute the state transition matrix \(\Phi(t, 0)\)
    \item Find \(P(t)\) such that \( z(t) = P(t)x(t) \).
          What property should \(P(t)\) satisfy to be a meaningful transformation matrix?
          Is it holding in this case?
    \item Let, \( \dot{x}(t)=A(t) x+B(t) u \) be a LTV system, and \( z(t)=P(t) x(t) \) is a transformation with \( P(t) \geq 0, \forall t \geq t_{0} \).
          Derive the LTV system that describes the dynamics of \( z \).
    \item Verify your relations in part (d) using \( A_{1}(t), A_{2}(t), P(t) \) of the above example.
\end{enumerate}

\subsection*{Solution}

\subsubsection*{(a) Two equivalent state-space representations}

Given the differential equation \( \ddot y - 2t\dot y - (2 - t^2)y = 0 \),
let \( x = \begin{bmatrix} x_1 \\ x_2 \end{bmatrix} = \begin{bmatrix} y \\ \dot y \end{bmatrix} \).
Then, we have
\begin{align*}
    \implies
    \dot x_1
     & =
    \dot y = x_2
    \\
    \implies
    \dot x_2
     & =
    \ddot y = 2t\dot y + (2 - t^2)y = 2tx_2 + (2 - t^2)x_1
\end{align*}
Therefore, the state-space representation of the system is given by
\begin{align*}
    \begin{bmatrix}
        \dot x_1 \\
        \dot x_2
    \end{bmatrix}
     & =
    \begin{bmatrix}
        0       & 1  \\
        2 - t^2 & 2t
    \end{bmatrix}
    \begin{bmatrix}
        x_1 \\
        x_2
    \end{bmatrix}
    \implies
    \dot x
    =
    \begin{bmatrix}
        0       & 1  \\
        2 - t^2 & 2t
    \end{bmatrix}
    x
    :=
    A_1(t) x
\end{align*}

Now, consider \( z = \begin{bmatrix} z_1 \\ z_2 \end{bmatrix} = \begin{bmatrix} y \\ -t y + \dot y \end{bmatrix} \).
(Note: it is easier to determine this by solving part (c) first).
Then, we have
\begin{align*}
    \implies
    \dot z_1
     & =
    \dot y = z_2 + t y
    =
    z_2 + t z_1
    \\
    \implies
    \dot z_2
     & =
    \frac{d}{dt} (-t y + \dot y)
    =
    (-t \dot y - y) + \ddot y
    \\ & =
    (-t \dot y - y) + \left( 2t \dot y + (2 - t^2)y \right)
    \\ & =
    t \dot y + y - t^2 y
    \\ & =
    t (z_2 + t z_1) + z_1 - t^2 z_1
    \\ & =
    t z_2 + \cancel{t^2 z_1} + z_1 - \cancel{t^2 z_1}
    \\ & =
    t z_2 + z_1
\end{align*}
Thereby, the state-space representation of the system is equivalently given by
\begin{align*}
    \begin{bmatrix}
        \dot z_1 \\
        \dot z_2
    \end{bmatrix}
     & =
    \begin{bmatrix}
        t & 1 \\
        1 & t
    \end{bmatrix}
    \begin{bmatrix}
        z_1 \\
        z_2
    \end{bmatrix}
    \implies
    \dot z
    =
    \begin{bmatrix}
        t & 1 \\
        1 & t
    \end{bmatrix}
    z
    :=
    A_2(t) z
\end{align*}

\subsubsection*{(b) State transition matrix}

For the system \( \dot z = A_2(t) z \), we can compute the state transition matrix, by first defining \( M(t) \) as
\begin{equation*}
    M(t)
    =
    \int_{0}^{t} A_2(\tau) d\tau
    =
    \int_{0}^{t} \begin{bmatrix}
        \tau & 1    \\
        1    & \tau
    \end{bmatrix}
    d\tau
    =
    \begin{bmatrix}
        \int_{0}^{t} \tau d\tau & \int_{0}^{t} d\tau      \\
        \int_{0}^{t} d\tau      & \int_{0}^{t} \tau d\tau
    \end{bmatrix}
    =
    \begin{bmatrix}
        \frac{t^2}{2} & t             \\
        t             & \frac{t^2}{2}
    \end{bmatrix}
\end{equation*}
Checking for the commutativity of \( M(t) \) and \( A_2(t) \) as follows,
\begin{align*}
    A_2(t) M(t)
     & =
    \begin{bmatrix}
        t & 1 \\
        1 & t
    \end{bmatrix}
    \begin{bmatrix}
        \frac{t^2}{2} & t             \\
        t             & \frac{t^2}{2}
    \end{bmatrix}
    =
    \begin{bmatrix}
        \frac{t^3}{2} + t & \frac{3t^2}{2}    \\
        \frac{3t^2}{2}    & \frac{t^3}{2} + t
    \end{bmatrix}
    \\
    M(t) A_2(t)
     & =
    \begin{bmatrix}
        \frac{t^2}{2} & t             \\
        t             & \frac{t^2}{2}
    \end{bmatrix}
    \begin{bmatrix}
        t & 1 \\
        1 & t
    \end{bmatrix}
    =
    \begin{bmatrix}
        \frac{t^3}{2} + t & \frac{3t^2}{2}    \\
        \frac{3t^2}{2}    & \frac{t^3}{2} + t
    \end{bmatrix}
\end{align*}
Thereby, \( M(t) \) and \( A_2(t) \) commute, and the state transition matrix is given by
\begin{equation*}
    \Phi(t, 0)
    =
    e^{M(t)}
    =
    I + M(t) + \frac{1}{2!} {(M(t))}^2 + \frac{1}{3!} {(M(t))}^3 + \ldots
    =
    \sum_{k=0}^{\infty} \frac{1}{k!} {(M(t))}^k
\end{equation*}
We can diagonalise \( M(t) \) as follows.
The eigenvalues can be found by solving the characteristic equation \( \text{det}(M(t) - \lambda I) = 0 \), as
\begin{align*}
    \text{det}(M(t) - \lambda I)
     & =
    \text{det}\left(
    \begin{bmatrix}
            \frac{t^2}{2} - \lambda & t                       \\
            t                       & \frac{t^2}{2} - \lambda
        \end{bmatrix}
    \right)
    =
    \left( \frac{t^2}{2} - \lambda \right)^2 - t^2
    \\ & =
    \lambda^2 - t^2 \lambda + \left(\frac{t^4}{4} - t^2\right)
    =
    0
    \implies
    \lambda
    =
    \frac{t^2}{2} \pm t
\end{align*}
The eigenvectors corresponding to the eigenvalues \( \frac{t^2}{2} + t \) and \( \frac{t^2}{2} - t \) can be calculated to be \( \begin{bmatrix} 1 \\ 1 \end{bmatrix} \) and \( \begin{bmatrix} 1 \\ -1 \end{bmatrix} \) respectively, giving
\begin{equation*}
    P = \begin{bmatrix}
        1 & 1  \\
        1 & -1
    \end{bmatrix}
    \implies
    P^{-1} = \frac{1}{2} \begin{bmatrix}
        1 & 1  \\
        1 & -1
    \end{bmatrix}
\end{equation*}
Thereby, the matrix \( M(t) \) can be diagonalised as
\begin{equation*}
    M(t)
    =
    PDP^{-1}
    =
    P
    \begin{bmatrix}
        \frac{t^2}{2} + t & 0                 \\
        0                 & \frac{t^2}{2} - t
    \end{bmatrix}
    P^{-1}
\end{equation*}
From this, we can compute \( e^{M(t)} \) as
\begin{align*}
    e^{M(t)}
     & =
    \sum_{k=0}^{\infty} \frac{1}{k!} {(M(t))}^k
    =
    \sum_{k=0}^{\infty} \frac{1}{k!} P D^k P^{-1}
    =
    P \left( \sum_{k=0}^{\infty} \frac{1}{k!} D^k \right) P^{-1}
    =
    P e^D P^{-1}
    \\
     & =
    \begin{bmatrix}
        1 & 1  \\
        1 & -1
    \end{bmatrix}
    \begin{bmatrix}
        e^{\frac{t^2}{2} + t} & 0                     \\
        0                     & e^{\frac{t^2}{2} - t}
    \end{bmatrix}
    \frac{1}{2}
    \begin{bmatrix}
        1 & 1  \\
        1 & -1
    \end{bmatrix}
    \\
     & =
    \frac{1}{2}
    \begin{bmatrix}
        e^{\frac{t^2}{2} + t} + e^{\frac{t^2}{2} - t} & e^{\frac{t^2}{2} + t} - e^{\frac{t^2}{2} - t} \\
        e^{\frac{t^2}{2} + t} - e^{\frac{t^2}{2} - t} & e^{\frac{t^2}{2} + t} + e^{\frac{t^2}{2} - t}
    \end{bmatrix}
    =
    \frac{1}{2} e^{\frac{t^2}{2}}
    \begin{bmatrix}
        (e^t + e^{-t}) & (e^t - e^{-t}) \\
        (e^t - e^{-t}) & (e^t + e^{-t})
    \end{bmatrix}
    \\
     & =
    e^{\frac{t^2}{2}}
    \begin{bmatrix}
        \cosh t & \sinh t \\
        \sinh t & \cosh t
    \end{bmatrix}
\end{align*}
Thereby, the state transition is given by
\begin{equation*}
    \boxed{
        \Phi(t, 0)
        =
        e^{\frac{t^2}{2}}
        \begin{bmatrix}
            \cosh t & \sinh t \\
            \sinh t & \cosh t
        \end{bmatrix}
    }
\end{equation*}

\subsubsection*{(c) Transformation matrix \( P(t) \)}

Given the transformation matrix given by \( z(t) = P(t)x(t) \), and the expressions for \( x(t) \) and \( z(t) \) from section (a), we have
\begin{align*}
    \begin{bmatrix}
        z_1 \\
        z_2
    \end{bmatrix}
     & =
    \begin{bmatrix}
        p_{11} & p_{12} \\
        p_{21} & p_{22}
    \end{bmatrix}
    \begin{bmatrix}
        x_1 \\
        x_2
    \end{bmatrix}
    =
    \begin{bmatrix}
        p_{11} x_1 + p_{12} x_2 \\
        p_{21} x_1 + p_{22} x_2
    \end{bmatrix}
    =
    \begin{bmatrix}
        p_{11} y + p_{12} \dot y \\
        p_{21} y + p_{22} \dot y
    \end{bmatrix}
    \\
    \implies
    z_1
     & =
    y
    =
    p_{11} y + p_{12} \dot y
    \implies
    p_{11} = 1, \quad p_{12} = 0
    \\
    \implies
    z_2
     & =
    -t y + \dot y
    =
    p_{21} y + p_{22} \dot y
    \implies
    p_{21} = -t, \quad p_{22} = 1
\end{align*}
Thereby, we get the transformation matrix as
\begin{equation*}
    \boxed{
        P(t)
        =
        \begin{bmatrix}
            1  & 0 \\
            -t & 1
        \end{bmatrix}
    }
\end{equation*}

\subsubsection*{(d) LTV system for \( z \)}

Given the LTV system \( \dot{x}(t)=A(t) x+B(t) u \) and the transformation \( z(t)=P(t) x(t) \), with \( P(t) \geq 0, \forall t \geq t_{0} \), we have
\begin{align*}
    \dot{z}(t)
     & =
    \frac{d}{dt} z(t)
    =
    \frac{d}{dt} (P(t) x(t))
    =
    \frac{d}{dt} \Big(P(t)\Big) x(t) + P(t) \frac{d}{dt} x(t)
    \\
     & =
    \dot{P}(t) x(t) + P(t) \Big( A(t) x(t) + B(t) u(t) \Big)
    \\ & =
    \Big( \dot{P}(t) + P(t) A(t) \Big) x(t) + P(t) B(t) u(t)
\end{align*}

For an invertible matrix \( P(t) \), we can write \( x(t) = P^{-1}(t) z(t) \), thereby giving the LTV system for \( z \) as
\[
    \boxed{
        \begin{aligned}
            \dot{z}(t)
             & =
            \tilde{A}(t) z(t) + \tilde{B}(t) u(t), \quad \text{where}
            \\
            \tilde{A}(t)
             & =
            \dot{P}(t) P^{-1}(t) + P(t) A(t) P^{-1}(t)
            \\
            \tilde{B}(t)
             & =
            P(t) B(t)
        \end{aligned}
    }
\]

\subsubsection*{(e) Verification}

For the given system, we have
\[
    A_1(t) =
    \begin{bmatrix}
        0       & 1  \\
        2 - t^2 & 2t
    \end{bmatrix}, \quad
    A_2(t) =
    \begin{bmatrix}
        t & 1 \\
        1 & t
    \end{bmatrix}, \quad
    P(t) =
    \begin{bmatrix}
        1  & 0 \\
        -t & 1
    \end{bmatrix}
\]
\begin{align*}
    \implies
    \dot{P}(t)
     & =
    \begin{bmatrix}
        0  & 0 \\
        -1 & 0
    \end{bmatrix}
    , \quad
    P^{-1}(t)
    =
    \begin{bmatrix}
        1 & 0 \\
        t & 1
    \end{bmatrix}
    \\
    \implies
    \tilde{A}(t)
     & =
    \begin{bmatrix}
        0  & 0 \\
        -1 & 0
    \end{bmatrix}
    \begin{bmatrix}
        1 & 0 \\
        t & 1
    \end{bmatrix}
    +
    \begin{bmatrix}
        1  & 0 \\
        -t & 1
    \end{bmatrix}
    \begin{bmatrix}
        0       & 1  \\
        2 - t^2 & 2t
    \end{bmatrix}
    \begin{bmatrix}
        1 & 0 \\
        t & 1
    \end{bmatrix}
    \\ & =
    \begin{bmatrix}
        0  & 0 \\
        -1 & 0
    \end{bmatrix}
    +
    \begin{bmatrix}
        1  & 0 \\
        -t & 1
    \end{bmatrix}
    \begin{bmatrix}
        t                & 1  \\
        (2 - t^2 + 2t^2) & 2t
    \end{bmatrix}
    \\ & =
    \begin{bmatrix}
        0  & 0 \\
        -1 & 0
    \end{bmatrix}
    +
    \begin{bmatrix}
        t                                                  & 1        \\
        (-\cancel{t^2} + 2 - \cancel{t^2} + \cancel{2t^2}) & (2t - t)
    \end{bmatrix}
    \\ & =
    \begin{bmatrix}
        0  & 0 \\
        -1 & 0
    \end{bmatrix}
    +
    \begin{bmatrix}
        t & 1 \\
        2 & t
    \end{bmatrix}
    =
    \begin{bmatrix}
        t & 1 \\
        1 & t
    \end{bmatrix}
    =
    A_2(t)
\end{align*}
as required.
Hence, it stands verified.
