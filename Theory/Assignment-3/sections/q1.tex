\section*{Problem 1}

Consider a system is modeled by the following differential equation:
\[
    \ddot y - 2t\dot y - (2 - t^2)y = 0
\]
\begin{enumerate}[label= (\alph*)]
    \item Show that the following are two equivalent state-space representations of the above system:
          \begin{align*}
              \dot x
               & =
              \begin{bmatrix}
                  0       & 1  \\
                  2 - t^2 & 2t
              \end{bmatrix}
              x
              :=
              A_1(t) x
              \\
              \dot z
               & =
              \begin{bmatrix}
                  t & 1 \\
                  1 & t
              \end{bmatrix}
              z
              :=
              A_2(t) z
          \end{align*}
          \textit{Hint:} \(x\) and \(z\) share a common component.
    \item Using the easiest choice, compute the state transition matrix \(\Phi(t, 0)\)
    \item Find \(P(t)\) such that \( z(t) = P(t)x(t) \).
          What property should \(P(t)\) satisfy to be a meaningful transformation matrix?
          Is it holding in this case?
    \item Let, \( \dot{x}(t)=A(t) x+B(t) u \) be a LTV system, and \( z(t)=P(t) x(t) \) is a transformation with \( P(t) \geq 0, \forall t \geq t_{0} \).
          Derive the LTV system that describes the dynamics of \( z \).
    \item Verify your relations in part (d) using \( A_{1}(t), A_{2}(t), P(t) \) of the above example.
\end{enumerate}

\subsection*{Solution}

\subsubsection*{(a) Two equivalent state-space representations}

Given the differential equation \( \ddot y - 2t\dot y - (2 - t^2)y = 0 \),
let \( x = \begin{bmatrix} x_1 \\ x_2 \end{bmatrix} = \begin{bmatrix} y \\ \dot y \end{bmatrix} \).
Then, we have
\begin{align*}
    \implies
    \dot x_1
     & =
    \dot y = x_2
    \\
    \implies
    \dot x_2
     & =
    \ddot y = 2t\dot y + (2 - t^2)y = 2tx_2 + (2 - t^2)x_1
\end{align*}
Therefore, the state-space representation of the system is given by
\begin{align*}
    \begin{bmatrix}
        \dot x_1 \\
        \dot x_2
    \end{bmatrix}
     & =
    \begin{bmatrix}
        0       & 1  \\
        2 - t^2 & 2t
    \end{bmatrix}
    \begin{bmatrix}
        x_1 \\
        x_2
    \end{bmatrix}
    \implies
    \dot x
    =
    \begin{bmatrix}
        0       & 1  \\
        2 - t^2 & 2t
    \end{bmatrix}
    x
    :=
    A_1(t) x
\end{align*}

Now, consider \( z = \begin{bmatrix} z_1 \\ z_2 \end{bmatrix} = \begin{bmatrix} y \\ -t y + \dot y \end{bmatrix} \).
Then, we have
\begin{align*}
    \implies
    \dot z_1
     & =
    \dot y = z_2 + t y
    =
    z_2 + t z_1
    \\
    \implies
    \dot z_2
     & =
    \frac{d}{dt} (-t y + \dot y)
    =
    (-t \dot y - y) + \ddot y
    \\ & =
    (-t \dot y - y) + \left( 2t \dot y + (2 - t^2)y \right)
    \\ & =
    t \dot y + y - t^2 y
    \\ & =
    t (z_2 + t z_1) + z_1 - t^2 z_1
    \\ & =
    t z_2 + \cancel{t^2 z_1} + z_1 - \cancel{t^2 z_1}
    \\ & =
    t z_2 + z_1
\end{align*}
Thereby, the state-space representation of the system is equivalently given by
\begin{align*}
    \begin{bmatrix}
        \dot z_1 \\
        \dot z_2
    \end{bmatrix}
     & =
    \begin{bmatrix}
        t & 1 \\
        1 & t
    \end{bmatrix}
    \begin{bmatrix}
        z_1 \\
        z_2
    \end{bmatrix}
    \implies
    \dot z
    =
    \begin{bmatrix}
        t & 1 \\
        1 & t
    \end{bmatrix}
    z
    :=
    A_2(t) z
\end{align*}

\subsubsection*{(b) State transition matrix}

\subsubsection*{(c) Transformation matrix \( P(t) \)}

Given the transformation matrix given by \( z(t) = P(t)x(t) \), and the expressions for \( x(t) \) and \( z(t) \) from section (a), we have
\begin{align*}
    \begin{bmatrix}
        z_1 \\
        z_2
    \end{bmatrix}
     & =
    \begin{bmatrix}
        p_{11} & p_{12} \\
        p_{21} & p_{22}
    \end{bmatrix}
    \begin{bmatrix}
        x_1 \\
        x_2
    \end{bmatrix}
    =
    \begin{bmatrix}
        p_{11} x_1 + p_{12} x_2 \\
        p_{21} x_1 + p_{22} x_2
    \end{bmatrix}
    =
    \begin{bmatrix}
        p_{11} y + p_{12} \dot y \\
        p_{21} y + p_{22} \dot y
    \end{bmatrix}
    \\
    \implies
    z_1
     & =
    y
    =
    p_{11} y + p_{12} \dot y
    \implies
    p_{11} = 1, \quad p_{12} = 0
    \\
    \implies
    z_2
     & =
    -t y + \dot y
    =
    p_{21} y + p_{22} \dot y
    \implies
    p_{21} = -t, \quad p_{22} = 1
\end{align*}
Thereby, we get the transformation matrix as
\begin{equation*}
    \boxed{
        P(t)
        =
        \begin{bmatrix}
            1  & 0 \\
            -t & 1
        \end{bmatrix}
    }
\end{equation*}
