\section*{Problem 4}

Consider the system \( x(k+1)=A x(k)+B u(k) \) with \( x(0)=x_{0}, k \geq 0 \).
Determine conditions under which there exists a sequence of inputs so that the state remains at \( x_{0} \), i.e., so that \( x(k)=x_{0} \) for all \( k \geq 0 \).
How is this input sequence determined?
Apply your method to the specific case:
\[
    A=\left[\begin{array}{cc}
            2 & 0  \\
            0 & -1
        \end{array}\right], B=\left[\begin{array}{l}
            1 \\
            1
        \end{array}\right], x_{0}=\left[\begin{array}{c}
            -2 \\
            1
        \end{array}\right]
\]

\subsection*{Solution}

The state equation is given by \( x(k+1)=A x(k)+B u(k) \).
We can see that
\begin{align*}
    x(1)
     & =
    A x(0)+B u(0)
    \\
    x(2)
     & =
    A x(1)+B u(1)
    =
    A^{2} x(0)+A B u(0)+B u(1)
    \\
    x(3)
     & =
    A x(2)+B u(2)
    =
    A^{3} x(0)+A^{2} B u(0)+A B u(1)+B u(2)
    \\
     & \vdots
    \\
    x(k + 1)
     & =
    A^{k + 1} x(0)+A^{k} B u(0)+A^{k - 1} B u(1)+\cdots+B u(k)
    \\ & =
    A^{k + 1} x(0)+\sum_{i=0}^{k} A^{k - i} B u(i)
\end{align*}
Given that \( x(k)=x_{0} \; \forall k \geq 0 \), we get
\begin{align*}
    x_0
    =
    A^{k + 1} x_0 + \sum_{i=0}^{k} A^{k - i} B u(i)
    \implies
    \sum_{i=0}^{k} A^{k - i} B u(i) = (I - A^{k + 1}) x_0, \quad \forall k \geq 0
\end{align*}
Thereby, the sequence of inputs \( u(i) \) is determined from
\begin{equation*}
    \boxed{
        \sum_{i=0}^{k} A^{k - i} B u(i) = (I - A^{k + 1}) x_0, \quad \forall k \geq 0
    }
\end{equation*}

For the given system, we have
\[
    A=\left[\begin{array}{cc}
            2 & 0  \\
            0 & -1
        \end{array}\right], B=\left[\begin{array}{l}
            1 \\
            1
        \end{array}\right], x_{0}=\left[\begin{array}{c}
            -2 \\
            1
        \end{array}\right]
\]
\begin{align*}
    \implies
    I - A^{k + 1}
     & =
    \left[\begin{array}{cc}
                  1 & 0 \\
                  0 & 1
              \end{array}\right]
    -
    \left[\begin{array}{cc}
                  2^{k + 1} & 0            \\
                  0         & (-1)^{k + 1}
              \end{array}\right]
    =
    \left[\begin{array}{cc}
                  1 - 2^{k + 1} & 0                \\
                  0             & 1 - (-1)^{k + 1}
              \end{array}\right]
    \\
    \implies
    (I - A^{k + 1}) x_0
     & =
    \left[\begin{array}{cc}
                  1 - 2^{k + 1} & 0                \\
                  0             & 1 - (-1)^{k + 1}
              \end{array}\right]
    \left[\begin{array}{c}
                  -2 \\
                  1
              \end{array}\right]
    =
    \left[\begin{array}{c}
                  -2(1 - 2^{k + 1}) \\
                  1 - (-1)^{k + 1}
              \end{array}\right]
\end{align*}
For \( k = 0 \), we can evaluate \( u(0) \) as
\begin{align*}
    B u(0)
     & =
    \left[\begin{array}{c}
                  1 \\
                  1
              \end{array}\right]
    u(0)
    =
    (I - A) x_0
    =
    \left[\begin{array}{c}
                  -2(1 - 2^{0 + 1}) \\
                  1 - (-1)^{0 + 1}
              \end{array}\right]
    =
    \left[\begin{array}{c}
                  2 \\
                  2
              \end{array}\right]
    \implies
    u(0)
    =
    2
\end{align*}
For \( k = 1 \), we can evaluate \( u(1) \) as
\begin{align*}
    A B u(0) + B u(1)
     & =
    \left[\begin{array}{cc}
                  2 & 0  \\
                  0 & -1
              \end{array}\right]
    \left[\begin{array}{c}
                  1 \\
                  1
              \end{array}\right]
    2
    +
    \left[\begin{array}{c}
                  1 \\
                  1
              \end{array}\right]
    u(1)
    =
    \left[\begin{array}{c}
                  4 \\
                  -2
              \end{array}\right]
    +
    \left[\begin{array}{c}
                  1 \\
                  1
              \end{array}\right]
    u(1)
    \\ & =
    (I - A^2) x_0
    =
    \left[\begin{array}{c}
                  -2(1 - 2^{1 + 1}) \\
                  1 - (-1)^{1 + 1}
              \end{array}\right]
    =
    \left[\begin{array}{c}
                  6 \\
                  0
              \end{array}\right]
    \\
    \implies
    \left[\begin{array}{c}
                  1 \\
                  1
              \end{array}\right]
    u(1)
     & =
    \left[\begin{array}{c}
                  6 \\
                  0
              \end{array}\right]
    -
    \left[\begin{array}{c}
                  4 \\
                  -2
              \end{array}\right]
    =
    \left[\begin{array}{c}
                  2 \\
                  2
              \end{array}\right]
    \implies
    u(1)
    =
    2
\end{align*}
We can similarly evaluate \( u(2), u(3), \ldots \) and so on.

Therefore, the sequence of inputs \( u(k) \) is given by
\[
    \boxed{
        u(k) = 2, \quad \forall k \geq 0
    }
\]

For evaluating the expression at \( k \), we can see that
\begin{align*}
    \sum_{i=0}^{k} A^{k - i} B u(i)
     & =
    \sum_{i=0}^{k}
    \left[\begin{array}{cc}
                  2^{k - i} & 0            \\
                  0         & (-1)^{k - i}
              \end{array}\right]
    \left[\begin{array}{c}
                  1 \\
                  1
              \end{array}\right]
    u(i)
    =
    \sum_{i=0}^{k}
    \left[\begin{array}{c}
                  2^{k - i} \\
                  (-1)^{k - i}
              \end{array}\right]
    u(i)
    \\ & =
    (I - A^{k + 1}) x_0
    =
    \left[\begin{array}{c}
                  -2(1 - 2^{k + 1}) \\
                  1 - (-1)^{k + 1}
              \end{array}\right]
    \\
    \implies
    \sum_{i=0}^{k}
    \left[\begin{array}{c}
                  2^{k - i} \\
                  (-1)^{k - i}
              \end{array}\right]
    u(i)
     & =
    \left[\begin{array}{c}
                  -2(1 - 2^{k + 1}) \\
                  1 - (-1)^{k + 1}
              \end{array}\right]
\end{align*}
which can then be solved recursively to determine the sequence of inputs \( u(k) \).
