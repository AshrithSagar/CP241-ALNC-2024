\section*{Problem 1}

Consider a second-order system is given by:
\begin{equation*}
    \dot{x}_{1}=-\frac{6 x_{1}}{u^{2}}+2 x_{2}, \quad \dot{x}_{2}=\frac{-2\left(x_{1}+x_{2}\right)}{u^{2}}
\end{equation*}
where \( u=1+x_{1}^{2} \). Let, \( V(x)=x_{1}^{2} /\left(1+x_{1}^{2}\right)+x_{2}^{2} \).
\begin{enumerate}[label= (\alph*)]
    \item Show that \( V(x)>0 \) and \( \dot{V}(x)<0 \) for all \( x \in \mathbb{R}^{2}-\{0\} \).
    \item Consider the hyperbola \( x_{2}=2 /\left(x_{1}-\sqrt{2}\right) \).
          By investigating the vector field on the boundary of this hyperbola, show that trajectories to the right of the branch in the first quadrant cannot cross that branch.
    \item Show that the origin is not globally asymptotically stable.
\end{enumerate}
\textit{Hint:} In part (b), show that \( \dot{x}_{2} / \dot{x}_{1}=-1 /\left(1+2 \sqrt{2} x_{1}+2 x_{1}^{2}\right) \) on the hyperbola, and compare with the slope of the tangents to the hyperbola.

\subsection*{Solution}

\subsubsection*{(a) \( V(x) > 0 \) and \( \dot{V}(x) < 0 \) for all \( x \in \mathbb{R}^2 - \{ 0 \} \)}

Given the function \( \displaystyle V(x) = \frac{x_1^2}{1 + x_1^2} + x_2^2 \), we can see that \( V(x) > 0 \) for all \( x \in \mathbb{R}^2 - \{ 0 \} \) as
\begin{align*}
     &
    x_1 \neq 0
    \implies
    x_1^2 > 0
    \implies
    \frac{x_1^2}{1 + x_1^2} > 0
    \\ &
    x_2 \neq 0
    \implies
    x_2^2 > 0
    \\ &
    \implies
    V(x) =
    \frac{x_1^2}{1 + x_1^2} + x_2^2
    > 0
\end{align*}
Now, we can see that
\begin{align*}
    \dot{V}(x)
     & =
    \frac{\partial V}{\partial x_1} \dot{x}_1 + \frac{\partial V}{\partial x_2} \dot{x}_2
    \\
    \implies
    \frac{\partial V}{\partial x_1}
     & =
    \frac{2 x_1 (1 + x_1^2) - x_1^2 (2 x_1)}{(1 + x_1^2)^2}
    =
    \frac{2 x_1 + \cancel{2 x_1^3} - \cancel{2 x_1^3}}{(1 + x_1^2)^2}
    =
    \frac{2 x_1}{(1 + x_1^2)^2}
    =
    \frac{2 x_1}{u^2}
    \\
    \implies
    \frac{\partial V}{\partial x_2}
     & =
    2 x_2
\end{align*}
\begin{align*}
    \implies
    \dot{V}(x)
     & =
    \frac{2 x_1}{u^2} \left( -\frac{6 x_1}{u^2} + 2 x_2 \right) + 2 x_2 \left( -\frac{2 (x_1 + x_2)}{u^2} \right)
    \\ & =
    \frac{2 x_1}{u^2} \left( -\frac{6 x_1}{u^2} + 2 x_2 \right) - \frac{4 x_2 (x_1 + x_2)}{u^2}
    \\ & =
    -\frac{12 x_1^2}{u^4} + \cancel{\frac{4 x_1 x_2}{u^2}} - \cancel{\frac{4 x_1 x_2}{u^2}} - \frac{4 x_2^2}{u^2}
    \\ & =
    \left( -12 \left( \frac{x_1}{u^2} \right)^2 - 4 \left( \frac{x_2}{u} \right)^2 \right)
    <
    0,
    \quad
    \forall x \in \mathbb{R}^2 - \{ 0 \}
\end{align*}
using the fact that \( -a^2 < 0 \) for all \( a \in \mathbb{R} - \{ 0 \} \).

Hence, \( V(x) > 0 \) and \( \dot{V}(x) < 0 \) for all \( x \in \mathbb{R}^2 - \{ 0 \} \).

\subsubsection*{(b) Trajectories to the right of the branch in the first quadrant for the hyperbola \( x_2 = 2 / (x_1 - \sqrt{2}) \) cannot cross that branch}

Given the hyperbola \( x_2 = 2 / (x_1 - \sqrt{2}) \), we can see that the slope of the tangent to the hyperbola is given by
\begin{align*}
    \frac{d x_2}{d x_1}
     & =
    \frac{d}{d x_1} \left( \frac{2}{x_1 - \sqrt{2}} \right)
    =
    \frac{d}{d x_1} \left( 2 (x_1 - \sqrt{2})^{-1} \right)
    =
    -2 (x_1 - \sqrt{2})^{-2}
    =
    -\frac{2}{(x_1 - \sqrt{2})^2}
\end{align*}
For the system, we can see that computing \( \frac{\dot{x}_2}{\dot{x}_1} \) corresponds to \( \frac{dx_2}{dt} \frac{dt}{d x_1} = \frac{dx_2}{d x_1} \) for the trajectories of the system.
\begin{align*}
    \implies
    \frac{\dot{x}_2}{\dot{x}_1}
     & =
    \frac{-2\left(x_{1}+x_{2}\right)}{\cancel{u^{2}}}
    \cdot
    \frac{\cancel{u^{2}}}{-6 x_{1}+2 x_{2} u^2}
    =
    \frac{\left(x_{1}+x_{2}\right)}{3 x_{1} - 2 x_{2} u^2}
\end{align*}
On the hyperbola, we can see this to be
\begin{align*}
    \frac{\left(x_{1}+x_{2}\right)}{3 x_{1} - 2 x_{2} u^2}
     & =
    \frac{x_1 + 2 / (x_1 - \sqrt{2})}{3 x_1 - 2 \left( 2 / (x_1 - \sqrt{2}) \right) (1+x_1^2)^2}
    \\ & =
    \frac{x_1 (x_1 - \sqrt{2}) + 2}{3 x_1 (x_1 - \sqrt{2}) - 4 (1+x_1^2)^2}
    =
    \frac{x_1^2 - \sqrt{2} x_1 + 2}{3 x_1^2 - 3 \sqrt{2} x_1 - 4 - x_1^4 - 8 x_1^2}
    \\ & =
    \frac{x_1^2 - \sqrt{2} x_1 + 2}{-x_1^4 - 5 x_1^2 - 3 \sqrt{2} x_1 - 4}
    =
    \frac{-1}{\left(1+2 \sqrt{2} x_{1}+2 x_{1}^{2}\right)}
    =
    -\frac{1}{(\sqrt{2} x_{1}+ 1)^2}
\end{align*}
Now, comparing the slopes, we find that the slope of the tangent to the hyperbola is always greater than the slope of the trajectories of the system, as seen in the plot in Figure~\ref{fig:q1b}.
Thereby, the trajectories to the right of the branch in the first quadrant cannot cross that branch.

\begin{figure}[htb]
    \centering
    \includegraphics[width=\textwidth]{figures/images/q1b.png}
    \caption{
        Vector field on the boundary of the hyperbola
    }\label{fig:q1b}
\end{figure}

\newpage
\subsubsection*{(c) The origin is not globally asymptotically stable}

Since the trajectories to the right of the branch in the first quadrant cannot cross that branch, we can see that if we start from any point in the first quadrant, we will never reach the origin.
Since there exists atleast one point which doesn't go to the origin as time goes to infinity, the origin is not globally asymptotically stable.
