\section*{Problem 1}

Consider a second-order system is given by:
\begin{equation*}
    \dot{x}_{1}=-\frac{6 x_{1}}{u^{2}}+2 x_{2}, \quad \dot{x}_{2}=\frac{-2\left(x_{1}+x_{2}\right)}{u^{2}}
\end{equation*}
where \( u=1+x_{1}^{2} \). Let, \( V(x)=x_{1}^{2} /\left(1+x_{1}^{2}\right)+x_{2}^{2} \).
\begin{enumerate}[label= (\alph*)]
    \item Show that \( V(x)>0 \) and \( \dot{V}(x)<0 \) for all \( x \in \mathbb{R}^{2}-\{0\} \).
    \item Consider the hyperbola \( x_{2}=2 /\left(x_{1}-\sqrt{2}\right) \).
          By investigating the vector field on the boundary of this hyperbola, show that trajectories to the right of the branch in the first quadrant cannot cross that branch.
    \item Show that the origin is not globally asymptotically stable.
\end{enumerate}
\textit{Hint:} In part (b), show that \( \dot{x}_{2} / \dot{x}_{1}=-1 /\left(1+2 \sqrt{2} x_{1}+2 x_{1}^{2}\right) \) on the hyperbola, and compare with the slope of the tangents to the hyperbola.

\subsection*{Solution}

\subsubsection*{(a) \( V(x) > 0 \) and \( \dot{V}(x) < 0 \) for all \( x \in \mathbb{R}^2 - \{ 0 \} \)}

Given the function \( \displaystyle V(x) = \frac{x_1^2}{1 + x_1^2} + x_2^2 \), we can see that \( V(x) > 0 \) for all \( x \in \mathbb{R}^2 - \{ 0 \} \) as
\begin{align*}
     &
    x_1 \neq 0
    \implies
    x_1^2 > 0
    \implies
    \frac{x_1^2}{1 + x_1^2} > 0
    \\ &
    x_2 \neq 0
    \implies
    x_2^2 > 0
    \\ &
    \implies
    V(x) =
    \frac{x_1^2}{1 + x_1^2} + x_2^2
    > 0
\end{align*}
Now, we can see that
\begin{align*}
    \dot{V}(x)
     & =
    \frac{\partial V}{\partial x_1} \dot{x}_1 + \frac{\partial V}{\partial x_2} \dot{x}_2
    \\
    \implies
    \frac{\partial V}{\partial x_1}
     & =
    \frac{2 x_1 (1 + x_1^2) - x_1^2 (2 x_1)}{(1 + x_1^2)^2}
    =
    \frac{2 x_1 + \cancel{2 x_1^3} - \cancel{2 x_1^3}}{(1 + x_1^2)^2}
    =
    \frac{2 x_1}{(1 + x_1^2)^2}
    =
    \frac{2 x_1}{u^2}
    \\
    \implies
    \frac{\partial V}{\partial x_2}
     & =
    2 x_2
\end{align*}
\begin{align*}
    \implies
    \dot{V}(x)
     & =
    \frac{2 x_1}{u^2} \left( -\frac{6 x_1}{u^2} + 2 x_2 \right) + 2 x_2 \left( -\frac{2 (x_1 + x_2)}{u^2} \right)
    \\ & =
    \frac{2 x_1}{u^2} \left( -\frac{6 x_1}{u^2} + 2 x_2 \right) - \frac{4 x_2 (x_1 + x_2)}{u^2}
    \\ & =
    -\frac{12 x_1^2}{u^4} + \cancel{\frac{4 x_1 x_2}{u^2}} - \cancel{\frac{4 x_1 x_2}{u^2}} - \frac{4 x_2^2}{u^2}
    \\ & =
    \left( -12 \left( \frac{x_1}{u^2} \right)^2 - 4 \left( \frac{x_2}{u} \right)^2 \right)
    <
    0,
    \quad
    \forall x \in \mathbb{R}^2 - \{ 0 \}
\end{align*}
using the fact that \( a^2 > 0 \) for all \( a \in \mathbb{R} - \{ 0 \} \).

Hence, \( V(x) > 0 \) and \( \dot{V}(x) < 0 \) for all \( x \in \mathbb{R}^2 - \{ 0 \} \).
