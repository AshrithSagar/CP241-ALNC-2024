\section*{Problem 2}

Prove the following result:
If the system \( \dot{x}=A x \) is Asymptotically Stable, then for any \( C<0 \), the unique solution to the matrix equation \( A^{\top} P+P A=C \) is given by:
\begin{equation*}\label{eq:1}
    P=-\int_{t=0}^{\infty} e^{A^{\top} t} C e^{A t} d t \tag{1}
\end{equation*}
Also, the solution \( P \) is positive definite and symmetric.
Show the proof systematically using the following steps:
\begin{enumerate}[label= (\alph*)]
    \item Show that the Integral~\ref{eq:1} is finite.
    \item Show that the Matrix P in~\ref{eq:1} satisfies Lyapunov Equation
    \item Show that the Matrix P in~\ref{eq:1} is symmetric.
    \item Show that the Matrix P in~\ref{eq:1} is positive definite.
    \item Show that, if \( \dot{x}=A x \) is asymptotically stable, then the Lyapunov equation has a unique solution. \\
          \textit{Hint:} Use contradiction method. Assume another solution \( \bar{P} \) and show \( \bar{P}=P \)
\end{enumerate}

\subsection*{Solution}

\subsubsection*{(a) The integral \( \displaystyle \int_{t=0}^{\infty} e^{A^{\top} t} C e^{A t} d t \) is finite}

We can see that the integral is for a matrix, and thereby, we consider bounds usings matrix norms, which are defined by
\begin{equation*}
    \left \| A \right \| = \max_{\left \| x \right \| = 1} \left \| A x \right \|
\end{equation*}
Now,
\begin{align*}
    \left \| e^{A^{\top} t} C e^{A t} \right \|
     & \leq
    \left \| e^{A^{\top} t} \right \| \left \| C \right \| \left \| e^{A t} \right \|
\end{align*}
by the properties of matrix norms.
Since \( A \) is asymptotically stable, we have \( \left \| e^{A t} \right \| \leq M e^{-\lambda t} \) for some \( M, \lambda > 0 \).
Thus, we have
\begin{align*}
    \int_{t=0}^{\infty} \left \| e^{A^{\top} t} C e^{A t} \right \| d t
     & \leq
    \int_{t=0}^{\infty} \left \| e^{A^{\top} t} \right \| \left \| C \right \| \left \| e^{A t} \right \| d t
    \\ & \leq
    \int_{t=0}^{\infty} M e^{-\lambda t} \left \| C \right \| M e^{-\lambda t} d t
    =
    M^2 \left \| C \right \| \int_{t=0}^{\infty} e^{-2 \lambda t} d t
    \\ & =
    M^2 \left \| C \right \| \left( \frac{1}{2 \lambda} \right)
    =
    \frac{M^2 \left \| C \right \|}{2 \lambda}
    <
    \infty
\end{align*}
Thus, the integral \( \displaystyle \int_{t=0}^{\infty} e^{A^{\top} t} C e^{A t} d t \) is finite.

\subsubsection*{(b) \( P \) satisfies the Lyapunov equation}

Given the system \( \dot x = A x \), the corresponding Lyapunov equation is given by
\begin{equation*}
    A^{\top} P + P A = C
\end{equation*}
where \( C < 0 \), i.e., \( C \) is a negative definite matrix.
We can substitute the value of \( P \) from the given equation~\eqref{eq:1} and verify if it satisfies the Lyapunov equation.
\begin{align*}
    A^\top P + P A
     & =
    A^\top \left( -\int_{t=0}^{\infty} e^{A^{\top} t} C e^{A t} d t \right) + \left( -\int_{t=0}^{\infty} e^{A^{\top} t} C e^{A t} d t \right) A
    \\ & =
    -\int_{t=0}^{\infty} A^\top e^{A^{\top} t} C e^{A t} d t - \int_{t=0}^{\infty} e^{A^{\top} t} C e^{A t} A d t
    \\ & =
    -\int_{t=0}^{\infty} \left( \left( \frac{d}{dt} e^{A^{\top} t} \right) C e^{A t} + e^{A^{\top} t} C \left( \frac{d}{dt} e^{A t} \right) \right) d t
    \\ & =
    -\int_{t=0}^{\infty} \frac{d}{dt} \left( e^{A^{\top} t} C e^{A t} \right) d t
    =
    - e^{A^{\top} t} C e^{A t} \Big|_{t=0}^{\infty}
    \\ & =
    - \lim_{t \to \infty} e^{A^{\top} t} C e^{A t} + e^{A^{\top} 0} C e^{A 0}
    =
    - 0 + C
    =
    C
\end{align*}
Thus, the matrix \underline{\( P \) satisfies the Lyapunov equation}.

\subsubsection*{(c) \( P \) is symmetric}

We can see that
\begin{align*}
    P^\top
     & =
    {\left( -\int_{t=0}^{\infty} e^{A^{\top} t} C e^{A t} d t \right)}^\top
    =
    -\int_{t=0}^{\infty} {\left( e^{A^{\top} t} C e^{A t} \right)}^\top d t
    \\ & =
    -\int_{t=0}^{\infty} {\left( e^{A t} \right)}^\top C^\top {\left( e^{A^{\top} t} \right)}^\top d t
    =
    -\int_{t=0}^{\infty} e^{A^\top t} C^\top e^{A t} d t
    \\ & =
    - \int_{t=0}^{\infty} e^{A^\top t} C e^{A t} d t
    =
    P
    \implies
    \boxed{
        P^\top = P
    }
\end{align*}
where we used the properties that \( {(AB)}^\top = B^\top A^\top \); \( {(A^\top)}^\top = A \); \( {(e^{A t})}^\top = e^{A^\top t} \), which can be obtained by expanding through the matrix exponential equation; and \( C^\top = C \), i.e., \( C \) is symmetric.
Thus, the matrix \underline{\( P \) is symmetric}.

\subsubsection*{(d) \( P \) is positive definite}

For \( P \) to be positive definite, we need to show that \( x^\top P x > 0 \) for all \( x \neq 0 \).
We can see that
\begin{align*}
    x^\top P x
     & =
    x^\top \left( -\int_{t=0}^{\infty} e^{A^{\top} t} C e^{A t} d t \right) x
    =
    -\int_{t=0}^{\infty} x^\top e^{A^{\top} t} C e^{A t} x d t
    \\ & =
    -\int_{t=0}^{\infty} {(e^{A t} x)}^\top C {(e^{A t} x)} d t
    =
    -\int_{t=0}^{\infty} y^\top C y d t
\end{align*}
where \( y = e^{A t} x \).
Since \( C < 0 \), i.e., \( C \) is negative definite, we have
\begin{align*}
    y^\top C y
     &
    \begin{cases}
        < 0 & \text{if } y \neq 0
        \\
        = 0 & \text{if } y = 0
    \end{cases}
\end{align*}
Now, we can see that \( x \neq 0 \iff y = e^{A t} x \neq 0 \; \forall t > 0 \).
Thus, we have
\begin{align*}
    x^\top P x
     &
    \begin{cases}
        > 0 & \text{if } x \neq 0
        \\
        = 0 & \text{if } x = 0
    \end{cases}
\end{align*}
thereby, \underline{\( P \) is positive definite}.

\subsubsection*{(e) The Lyapunov equation has a unique solution}

Let \( P \) and \( \bar{P} \) be two solutions to the Lyapunov equation, i.e., we have
\begin{equation*}
    A^\top P + P A = C,
    \quad
    A^\top \bar{P} + \bar{P} A = C
\end{equation*}
Then, we can see that
\begin{align*}
     &
    C
    =
    A^\top P + P A
    =
    A^\top \bar{P} + \bar{P} A
    \implies
    A^\top (P - \bar{P}) + (P - \bar{P}) A
    =
    0
    \\
    \implies
     &
    A^\top Q + Q A
    =
    0,
    \quad \text{where } Q = (P - \bar{P})
\end{align*}
Since the system \( \dot x = A x \) is asymptotically stable, if we consider \( V(x) = x^\top Q x \), then we can see that \( V(x) \) is a Lyapunov function, by
\begin{align*}
    \implies
    \dot V(x)
     & =
    \frac{d}{dt} \left( x^\top Q x \right)
    =
    \frac{d}{dt} \left( x^\top \right) Q x + x^\top Q \frac{d}{dt} \left( x \right)
    =
    {\left(\frac{d}{dt} x \right)}^\top Q x + x^\top Q {\left(\frac{d}{dt} x \right)}
    \\ & =
    {\dot x}^\top Q x + x^\top Q \dot x
    =
    {\left( A x \right)}^\top Q x + x^\top Q \left( A x \right)
    =
    \left( x^\top A^\top \right) Q x + x^\top Q \left( A x \right)
    \\ & =
    x^\top A^\top Q x + x^\top Q A x
    =
    x^\top \left( A^\top Q + Q A \right) x
    =
    x^\top \left( 0 \right) x
    =
    0
\end{align*}
