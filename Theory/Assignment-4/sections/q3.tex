\section*{Problem 3}

Investigate Input-to-State Stability for the following systems:
\begin{enumerate}[label= (\alph*)]
    \item \( \dot{x}=-(1+u) x^{3} \)
    \item \( \dot{x}_{1}=-x_{1}+x_{1}^{2} x_{2}, \quad \dot{x}_{2}=-x_{1}^{3}-x_{2}+u \)
    \item \( \dot{x}_{1}=-x_{1}-x_{2}+u_{1}, \quad \dot{x}_{2}=x_{1}-x_{2}^{3}+u_{2} \)
\end{enumerate}

\subsection*{Solution}

For the system to be Input-to-State Stable, it should satisfy
\begin{equation*}
    \left| x(t) \right| \leq \beta( \| x(0) \|, t ) + \gamma( \sup_{s \in [0, t]} \left| u(s) \right| )
\end{equation*}
for some class-\( \mathcal{KL} \) function \( \beta \) and class-\( \mathcal{K} \) function \( \gamma \).

\subsubsection*{(a) System:\( \quad \dot{x}=-(1+u) x^{3} \)}

Given the system \( \dot{x}=-(1+u) x^{3} \), we will first check if the corresponding system with zero input is globally asymptotically stable.
Consider the Lyapunov candidate function as \( V(x)=\frac{1}{2} x^{2} \).
We can see that \( V(x) > 0 \) for all \( x \in \mathbb{R} - \{ 0 \} \), since
\begin{equation*}
    x \neq 0
    \implies
    x^2 > 0
    \implies
    V(x)
    =
    \frac{1}{2} x^2 > 0
\end{equation*}
Now, we have,
\begin{equation*}
    \dot{V}(x)
    =
    \frac{\partial V}{\partial x} \dot{x}
    =
    x \dot{x}
    =
    -x (1 + u) x^3
    =
    -(1 + u) x^4
\end{equation*}
For the zero-input condition, we can see that \( \dot{V}(x) = -x^4 < 0 \) for all \( x \in \mathbb{R} - \{ 0 \} \), thereby, the system is globally asymptotically stable under the zero input condition.

\subsubsection*{(b) System:\( \quad \dot{x}_{1}=-x_{1}+x_{1}^{2} x_{2}, \quad \dot{x}_{2}=-x_{1}^{3}-x_{2}+u \)}

Given the system \( \dot{x}_{1}=-x_{1}+x_{1}^{2} x_{2} \) and \( \dot{x}_{2}=-x_{1}^{3}-x_{2}+u \), we will first check if the corresponding system with zero input is globally asymptotically stable.
Consider the Lyapunov candidate function as \( V(x)=\frac{1}{2} x_{1}^{2}+\frac{1}{2} x_{2}^{2} \).
We can see that \( V(x) > 0 \) for all \( x \in \mathbb{R}^2 - \{ 0 \} \), since
\begin{equation*}
    x_i \neq 0
    \implies
    x_i^2 > 0
    \implies
    \frac{1}{2} x_i^2 > 0,
    \;
    i = 1, 2
    \implies
    V(x)
    =
    \frac{1}{2} x_1^2 + \frac{1}{2} x_2^2
    > 0
\end{equation*}
Now, we have,
\begin{align*}
    \dot{V}(x)
     & =
    \frac{\partial V}{\partial x_1} \dot{x}_1 + \frac{\partial V}{\partial x_2} \dot{x}_2
    \\
     & =
    x_1 \left( -x_1 + x_1^2 x_2 \right) + x_2 \left( -x_1^3 - x_2 + u \right)
    \\
     & =
    -x_1^2 + \cancel{x_1^3 x_2} - \cancel{x_1^3 x_2} - x_2^2 + u x_2
    \\
     & =
    -x_1^2 - x_2^2 + u x_2
\end{align*}
For the zero-input condition, we can see that \( \dot{V}(x) = -x_1^2 - x_2^2 < 0 \) for all \( x \in \mathbb{R} - \{ 0 \} \), thereby, the system is globally asymptotically stable under the zero input condition.

Now, by Young's inequality, we can see that for any \( a, b \in \mathbb{R} \) and \( \epsilon > 0 \), we have
\begin{equation*}
    \left| a b \right|
    \leq
    \frac{1}{2 \epsilon} a^2 + \frac{\epsilon}{2} b^2
\end{equation*}
Putting \( a = u \), \( b = x_2 \) and \( \epsilon = 1 \), we have
\begin{equation*}
    u x_2 \leq \left| u x_2 \right|
    \leq
    \frac{1}{2} u^2 + \frac{1}{2} x_2^2
\end{equation*}
Now, we have
\begin{align*}
    \dot{V}(x)
     & =
    -x_1^2 - x_2^2 + u x_2
    \\ & \leq
    -x_1^2 - x_2^2 + \frac{1}{2} u^2 + \frac{1}{2} x_2^2
    \\ & =
    \left( -x_1^2 - \frac{1}{2} x_2^2 \right) + \left( \frac{1}{2} u^2 \right)
    =
    - \alpha_3(|x|) + \alpha_4(|u|)
\end{align*}
where \( \alpha_3(|x|) = x_1^2 + \frac{1}{2} x_2^2 \) and \( \alpha_4(|u|) = \frac{1}{2} u^2 \) are class-\( \mathcal{K} \) functions.

Thereby, the \underline{system is Input-to-State Stable}.

\subsubsection*{(c) System:\( \quad \dot{x}_{1}=-x_{1}-x_{2}+u_{1}, \quad \dot{x}_{2}=x_{1}-x_{2}^{3}+u_{2} \)}

Given the system \( \dot{x}_{1}=-x_{1}-x_{2}+u_{1} \) and \( \dot{x}_{2}=x_{1}-x_{2}^{3}+u_{2} \), we will first check if the corresponding system with zero input is globally asymptotically stable.
Consider the Lyapunov candidate function as \( V(x)=\frac{1}{2} x_{1}^{2}+\frac{1}{2} x_{2}^{2} \).
We can see that \( V(x) > 0 \) for all \( x \in \mathbb{R}^2 - \{ 0 \} \), since
\begin{equation*}
    x_i \neq 0
    \implies
    x_i^2 > 0
    \implies
    \frac{1}{2} x_i^2 > 0,
    \;
    i = 1, 2
    \implies
    V(x)
    =
    \frac{1}{2} x_1^2 + \frac{1}{2} x_2^2
    > 0
\end{equation*}
Now, we have,
\begin{align*}
    \dot{V}(x)
     & =
    \frac{\partial V}{\partial x_1} \dot{x}_1 + \frac{\partial V}{\partial x_2} \dot{x}_2
    \\
     & =
    x_1 \left( -x_1 - x_2 + u_1 \right) + x_2 \left( x_1 - x_2^3 + u_2 \right)
    \\
     & =
    -x_1^2 - \cancel{x_1 x_2} + u_1 x_1 + \cancel{x_1 x_2} - x_2^4 + u_2 x_2
\end{align*}
For the zero-input condition, we can see that \( \dot{V}(x) = -x_1^2 - x_2^4 < 0 \) for all \( x \in \mathbb{R} - \{ 0 \} \), thereby, the system is globally asymptotically stable under the zero input condition.
